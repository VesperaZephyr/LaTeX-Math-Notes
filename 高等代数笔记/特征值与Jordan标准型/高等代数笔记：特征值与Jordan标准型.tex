\documentclass[12pt, a4paper]{ctexart}
\usepackage{amsmath, amsthm, amssymb, appendix, bm, graphicx, hyperref, mathrsfs, geometry,tikz,tikz-cd,enumerate}
\everymath{\displaystyle}
\geometry{left=2cm,right=2cm,top=2cm,bottom=2cm}
\usepackage{indentfirst}
\setlength{\parindent}{2em}	%设置首行缩进
\usepackage{manfnt}
\usetikzlibrary{arrows.meta,decorations.markings}
\usepackage[lite,subscriptcorrection,slantedGreek,nofontinfo]{mtpro2}
\usepackage{fancyhdr} %设置页眉页脚
\renewcommand{\footrulewidth}{0.5pt}
\pagestyle{fancy}
\fancyhf{}	%清除页眉页脚R
\fancyfoot[C]{\thepage}
\fancyhead[L]{\kaishu \leftmark}
\fancyhead[C]{\fangsong 关注作者知乎\href{https://www.zhihu.com/people/11-68-36-12}{\heiti 晨锦辉永生之语}谢谢喵}
\fancyhead[R]{\textit\thepage}

\allowdisplaybreaks[1] %n的值为0到4,表示分页的坚决程度,例如0表示能不分页就不分页,4表示强制分页. 
\title{\textbf{高等代数笔记:特征值到标准型}}
\author{晨锦辉永生之语}
\date{\today}
\linespread{1.5}

\usepackage{tcolorbox}
\usepackage{xcolor}
\tcbuselibrary{most, skins, breakable, theorems}
\tcbset{colback = red!5!white, colframe = red!75!black}
\newenvironment{solution}{\begin{proof}[解]}{\end{proof}}
\newtheorem*{remark}{{\color{red}\dbend}\textbf{注记}}
\newtheorem{theorem}{\indent 定理}[subsection]
%skins 程序包提供了各种「皮肤」,可以在基础 tcolorbox 的基础上扩展更多样式:attach boxed title to top left 这一效果就来自 skins 程序包中的 enhanced 主题. breakable 的效果则使得文本框能够跨页. %
\newtcbtheorem[number within=subsection]{definition}{定义}%
{drop shadow southeast,enhanced, breakable, theorem hanging indent=-5mm, arc=0mm, colback=red!5!white,colframe=red!75!black,fonttitle=\bfseries}{def}
\newtcbtheorem[number within=subsection]{proposition}{命题}%
{drop shadow southeast,enhanced, breakable, theorem hanging indent=-5mm, arc=0mm, colframe=cyan!75!black,colback=cyan!10!white,fonttitle=\bfseries}{pro}
\newtcbtheorem[number within=section]{lemma}{引理}%
{drop shadow southeast, enhanced, breakable, theorem hanging indent=-5mm, arc=0mm, colframe=orange!75!black,colback=orange!10!white,fonttitle=\bfseries}{lem}
\newtcbtheorem[number within=section]{corollary}{推论}%
{drop shadow southeast, enhanced, breakable, theorem hanging indent=-5mm, arc=0mm, colframe=orange!75!black,colback=orange!10!white,fonttitle=\bfseries}{cor}
\newtcbtheorem[number within=subsection]{example}{例}%
{drop shadow southeast, enhanced, breakable, theorem hanging indent=-5mm, arc=0mm,coltitle=black,  colframe=yellow!70!white,colback=yellow!10!white,fonttitle=\bfseries}{ex}
\tcolorboxenvironment{proof}{% `proof' from `amsthm'
	blanker,breakable,left=5mm,
	before skip=10pt,after skip=10pt,
	borderline west={1mm}{0pt}{violet}}
\tcbset{pikachu/.style={enhanced,colback=yellow,colframe=black,boxrule=0.6mm,enlarge top by=7.0mm,enlarge bottom by=2.0mm,top=50pt,sharp corners=south,arc=14mm,
		overlay={
			\begin{scope}[shift={([xshift=9.0mm,yshift=-13mm]frame.north west)},rotate=30]
				% 左眼
				\path[draw=black,fill=black,line width=0.5mm] (0,0) arc (0:360:2mm);
				\path[fill=yellow] (-.05,.08) arc (0:360:1mm);
				% 右眼
				\path[draw=black,fill=black,line width=0.5mm] (1.2,0) arc (0:360:2mm);
				\path[fill=yellow] (1.1-.05,.08) arc (0:360:1mm);
				% ハナ
				\path[draw=black,fill=black] (0.4,-.15) circle [x radius=0.06,y radius=0.03] (0:360);
				% クチ
				\path[draw=black,line width=0.4mm,xshift=0.5mm,yshift=-3.5mm] (0,-.02) .. controls (.1,-.1) and (.15,-.14) .. (.35,0) % 左
				.. controls (-.15+.7,-.14) and (-.1+.7,-.1) .. (0+.7,-.02); % 右
				% ほっぺ
				\path[draw=black,fill=red,line width=0.4mm] (1.6,-0.4) arc (10:290:2mm);
		\end{scope}}
		,
		underlay={
			\begin{scope}[shift={([xshift=0mm,yshift=0mm]frame.north west)}]
				% 耳とフレームが重なるところを白塗り(右)
				\path[draw=yellow,line width=0.7mm] (1.51,-0.03)--(2.55,-0.03);
				% 耳とフレームが重なるところを白塗り(左)
				\path[draw=yellow,line width=2.0mm] (0.1,-0.84)--(0.1,-2);
		\end{scope}}
		,
		% 右耳
		underlay={
			\begin{scope}[shift={([xshift=0mm,yshift=0mm]frame.north west)}]
				% 耳のメイン
				\path[draw=black,fill=yellow,line width=0.6mm,rounded corners=1.0pt] (1.5,-0.03) .. controls (2.5,0.3) and (3.5,-0.5) .. (3.7,-0.6) .. controls (2.7,-0.5) and (2.5,-0.5) .. (2.2,-0.4);
				
				% 耳の黒い部分の境界
				\clip (1.5,-0.03) .. controls (2.5,0.3) and (3.5,-0.5) .. (3.7,-0.6) .. controls (2.7,-0.5) and (2.5,-0.5) .. (2.2,-0.4);
				
				\fill[black] (2.4,-0.5) to [out=10,in=210] (3.4,-0.3) -- (4,-0.7) -- cycle;
		\end{scope}}
		,
		% 左耳
		underlay={
			\begin{scope}[shift={([xshift=9.06mm,yshift=4.93mm]frame.north west)},rotate=60]
				% 耳のメイン xscale=-1 で反転
				\path[xscale=-1,draw=black,fill=yellow,line width=0.6mm,rounded corners=1.0pt] (1.5,-0.03) .. controls (2.5,0.3) and (3.5,-0.5) .. (3.7,-0.6) .. controls (2.7,-0.5) and (2.5,-0.5) .. (2.2,-0.4);
				
				% 耳の黒い部分の境界
				\clip[xscale=-1] (1.5,-0.03) .. controls (2.5,0.3) and (3.5,-0.5) .. (3.7,-0.6) .. controls (2.7,-0.5) and (2.5,-0.5) .. (2.2,-0.4);
				
				\fill[xscale=-1,black] (2.4,-0.5) to [out=10,in=210] (3.4,-0.3) -- (4,-0.7) -- cycle;
		\end{scope}}
		,
		% しっぽ
		underlay={
			\begin{scope}
				[xscale=1.1,yscale=0.4,shift={([xshift=-5mm,yshift=-19mm]frame.north east)},rotate=38]
				% [xscale=1,yscale=1,shift={([xshift=-8mm,yshift=-50mm]frame.north east)},rotate=0]
				% グリッド
				% \draw [help lines] (-6,0) grid (6,6);
				\path[draw=black,fill=yellow,line width=0.6mm,rounded corners=1.0pt]
				(0,0) -- (0.3,0) -- (0.7,1.2) -- (-0.5,1.4) -- (-0.1,2.7) -- (-1.8,3) to [out=80,in=245] (-1,5.4) -- (-3.9,6) to [out=245,in=90] (-4.6,2.2) -- (-2,2) -- (-2.2,1.1) -- (0.2,0.7) -- cycle;
				
				% 上とおなじの clip
				\clip (0,0) -- (0.3,0) -- (0.7,1.2) -- (-0.5,1.4) -- (-0.1,2.7) -- (-2,3) to [out=80,in=245] (-1.2,5.4) -- (-4.1,6) to [out=245,in=90] (-4.8,2.2) -- (-2.2,2) -- (-2.5,1.1) -- (0.2,0.7) -- cycle;
				\fill (-0.8,0.7) -- (-0.2,0.9) -- (-0.5,1.1) -- (-0.2,1.1) -- (-0.4,1.2) -- (-0.1,1.25) -- (-0.4,1.45)
				-- (1,1.5) -- (1,0) -- (-0.5,0) -- cycle;
		\end{scope}}
		,
		% 背中の模様
		underlay={
			\begin{scope}[shift={(frame.north west)},rounded corners=10pt]
				\path[fill=black,xshift=36mm,yshift=0mm] (0,0) -- (0.3,-0.8) -- (0.35,0);
				\path[fill=black,xshift=42mm,yshift=0mm] (0,0) -- (0.3,-0.8) -- (0.35,0);
		\end{scope}}
}}

	
\begin{document}

\maketitle
\tableofcontents
\newpage
\section{初看特征值}
\subsection{特征值}
给定线性空间 $V$ 上的线性变换 $\mathcal A$,我们想找到 $V$ 的一组基$\{\bm e_1, \bm e_2, \cdots, \bm e_n\}$ ,使
线性变换 $\mathcal A$ 在这组基下的表示矩阵为对角矩阵:
\[
\begin{pmatrix}
	a_1 & & & \\
	& a_2 & & \\
	& & \ddots & \\
	& & & a_n
\end{pmatrix}.
\]
这时,若 $\bm\alpha = k_1\bm e_1 + k_2\bm e_2 + \cdots + k_n\bm e_n$,则
\[
\mathcal A\bm\alpha = a_1k_1\bm e_1 + a_2k_2\bm e_2 + \cdots + a_nk_n\bm e_n.
\]
线性变换 $\mathcal A$ 的表达式非常简单,线性变换 $\mathcal A$ 的许多性质也变得一目了然. 例如,
若 $a_1, a_2, \cdots, a_r$ 不为零,而 $a_{r+1} = \cdots = a_n = 0$,则 $\mathcal A$ 的秩为 $r$,且 $\text{Im}\ \mathcal A$ 就是由
$\{\bm e_1, \bm e_2, \cdots, \bm e_r\}$ 生成的子空间,而 $\text{Ker}\ \mathcal A$ 则是由 $\{\bm e_{r+1}, \cdots, \bm e_n\}$ 生成的子空间. 

我们知道一个线性变换在不同基下的表示矩阵是相似的. 因此
用矩阵的语言重述上面提到的问题就是:能否找到一类特别简单的矩阵,使任一定
矩阵都与这类矩阵中的某一个相似?比如,我们可以问:是否所有的矩阵都相似于对角矩阵?若不然,哪一类矩阵可以相似于对角矩阵?

若线性空间 $V$ 可分解为\begin{equation}\label{oplus-1}
	V = V_1 \oplus V_2 \oplus \cdots \oplus V_m,
\end{equation}
其中每个 $V_i$ 都是线性变换 $\mathcal A$ 的不变子空间,那么 $\mathcal A$ 可以表示为分块对角阵. 我们希望 \eqref{oplus-1} 式中的 $V_i$ 越小越好. 最小的非零子空间是一维子空间. 若 $V_i$
是一维子空间,$x$ 是其中的任一非零向量,$\mathcal A$ 在 $V_i$ 上的作用相当于一个数乘,于是
存在 $\lambda_0 \in \mathbb{K}$,使
\[ \mathcal A(x) = \lambda_0 x. \]
\begin{definition}{特征值与特征向量}{eigenvalue}
	设 $\mathcal A$ 是数域 $\mathbb{K}$ 上线性空间 $V$ 上的线性变换,若 $\lambda_0 \in \mathbb{K}$,$x \in V$
	且 $x \neq 0$,使\begin{equation}\label{eigenvalue}
		\mathcal A(x) = \lambda_0 x,
	\end{equation}
	则称 $\lambda_0$ 是线性变换 $\mathcal A$ 的一个特征值,向量 $x$ 称为 $\mathcal A$ 关于特征值 $\lambda_0$ 的特征向量. 
\end{definition}
现在设 $\mathcal A$ 在某组基下的表示矩阵为 $\bm A$,向量 $\bm x$ 在这组基下可表示为一个列向量 $\bm\alpha$,这时\eqref{eigenvalue}式等价于
\begin{equation}
	\bm{A\alpha} = \lambda_0 \bm \alpha\iff(\lambda_0\bm I_n - \bm A)\bm\alpha = 0.
\end{equation}
因此,类似线性变换,我们可以定义矩阵的特征值、特征向量、特征子空间. 
\begin{definition}{矩阵的特征值、特征向量和特征子空间}{}
	设 $\bm A$ 是数域 $\mathbb{K}$ 上的 $n$ 阶方阵,若存在 $\lambda_0 \in \mathbb{K}$ 及 $n$ 维非零列向
	量 $\bm\alpha$,使$\bm{A\alpha} = \lambda_0$成立,则称 $\lambda_0$ 为矩阵 $A$ 的一个\textbf{特征值},$\bm\alpha$ 为 $A$ 关于特征值 $\lambda_0$
	的\textbf{特征向量}. 齐次线性方程组 $(\lambda_0\bm I_n -\bm A)\bm x = 0$ 的解空间 $V_{\lambda_0}$ 称为 $A$ 关于特征值
	$\lambda_0$ 的\textbf{特征子空间}. 
\end{definition}
\begin{proposition}{特征子空间}{}
	$\mathcal A$ 关于特征值 $\lambda_0$ 的全体特征向量再加上零向量构成 $V$ 的一个子空间. 
\end{proposition}
\begin{proof}
	若向量 $\bm x, \bm y$ 是关于特征值 $\lambda_0$ 的特征向量,则
	\[ \mathcal A(\bm x + \bm y) = \mathcal A(\bm x) + \mathcal A(\bm y) = \lambda_0\bm x + \lambda_0\bm y = \lambda_0 (\bm x + \bm y), \]
	\[ \mathcal A(c\bm x) = c\mathcal A(\bm x) = c\lambda_0\bm x = \lambda_0 (c\bm x). \]
	因此 $\mathcal A$ 的关于特征值 $\lambda_0$ 的全体特征向量加上零向量构成 $V$ 的子空间,记为 $V_{\lambda_0}$,
	称为 $\mathcal A$ 的关于特征值 $\lambda_0$ 的\textbf{特征子空间}.
\end{proof}
显然 $V_{\lambda_0}$ 是 $\mathcal A$ 的不变子空间. 

我们已经定义了线性变换与矩阵的特征值,现在的问题是如果来求一个线性
变换或一个矩阵的特征值?从 (6.1.4) 式可以看出,要使 $\alpha$ 非零,必须 $|\lambda_0 I_n - A| = 0$. 
反过来,若 $\lambda_0 \in \mathbb{K}$ 且 $|\lambda_0 I_n - A| = 0$,则 (6.1.4) 式有非零解 $\alpha$. 因此寻找矩阵 $A$ 的特征值等价于寻找行列式 $|\lambda\bm{I}_n - A| = 0$ 时 $\lambda$ 的值. 设 $A = (a_{ij})$,则
\[ |\lambda\bm{I}_n - A| = \begin{vmatrix}
	\lambda - a_{11} & -a_{12} & \cdots & -a_{1n} \\
	-a_{21} & \lambda - a_{22} & \cdots & -a_{2n} \\
	\vdots & \vdots & & \vdots \\
	-a_{n1} & -a_{n2} & \cdots & \lambda - a_{nn}
\end{vmatrix} \tag{6.1.5}
\]
是一个以 $\lambda$ 为未知数的 $n$ 次首一多项式. 
\begin{definition}{特征多项式}{}
	设 $\bm A$ 是 $n$ 阶方阵,称 $|\lambda \bm I_n - \bm A|$ 为 $A$ 的特征多项式. 
\end{definition}
由上面的讨论可得矩阵 $A$ 的特征值就是它的特征多项式的根. 
\begin{proposition}{}{}
	设 $ \bm A $ 是数域 $ K $ 上的 $ n $ 级矩阵,则 $ \bm A $ 的特征多项式 $ |\lambda\bm{I} - \bm A| $ 是一个 $ n $ 次多项式,$ \lambda^n $ 的系数是 1,$ \lambda^{n-1} $ 的系数等于 $ -\text{tr}(\bm A) $,常数项为 $ (-1)^n |\bm A| $,$ \lambda^{n-k} $ 的系数为 $ \bm A $ 的所有 $ k $ 阶主子式的和乘以 $ (-1)^k $,$ 1 \leqslant k < n $. 
\end{proposition}
\begin{proof}
	设 $ \bm A = (a_{ij}) $ 的列向量组是 $ \boldsymbol{a}_1, \boldsymbol{a}_2, \cdots, \boldsymbol{a}_n $.
	\[
	|\lambda\bm{I} - \bm A| = \begin{vmatrix}
		\lambda - a_{11} & 0 - a_{12} & \cdots & 0 - a_{1n} \\
		0 - a_{21} & \lambda - a_{22} & \cdots & 0 - a_{2n} \\
		\vdots & \vdots & \ddots & \vdots \\
		0 - a_{n1} & 0 - a_{n2} & \cdots & 0 - a_{nn}
	\end{vmatrix}
	\]
	利用行列式的性质,$ |\lambda\bm{I} - \bm A| $ 可以拆成 $ 2^n $ 个行列式的和,它们是
	\[
	\begin{vmatrix}
		\lambda & 0 & \cdots & 0 \\
		0 & \lambda & \cdots & 0 \\
		\vdots & \vdots & \ddots & \vdots \\
		0 & 0 & \cdots & \lambda
	\end{vmatrix} \cdot \begin{vmatrix}
		-a_{11} & -a_{12} & \cdots & -a_{1n} \\
		-a_{21} & -a_{22} & \cdots & -a_{2n} \\
		\vdots & \vdots & \ddots & \vdots \\
		-a_{n1} & -a_{n2} & \cdots & -a_{nn}
	\end{vmatrix},
	\]
	
	\[
	|(-\boldsymbol{a}_1, \cdots, -\boldsymbol{a}_{j_1-1}, \lambda \boldsymbol{e}_{j_1}, -\boldsymbol{a}_{j_1+1}, \cdots, \lambda \boldsymbol{e}_{j_2}, \cdots, -\boldsymbol{a}_n)|
	\]
	其中 $ 1 \leqslant j_1 < \cdots < j_{n-k} \leqslant n, k = 1, 2, \cdots, n-1 $.
	
	上述第 1 个行列式等于 $ \lambda^n $,第 2 个行列式等于 $ (-1)^k |\bm A| $,对于第 3 种类型的行列式,按第 $ j_1, j_2, \cdots, j_{n-k} $ 列展开,这 $ n-k $ 列元素组成的 $ n-k $ 阶子式只有一个不为 0:
	
	\[
	\begin{vmatrix}
		\lambda & 0 & \cdots & 0 \\
		0 & \lambda & \cdots & 0 \\
		\vdots & \vdots & \ddots & \vdots \\
		0 & 0 & \cdots & \lambda
	\end{vmatrix} = \lambda^{n-k},
	\]
	
	其余 $ n-k $ 阶子式全为 0. 这个不等于 0 的 $ n-k $ 阶子式的代数余子式为
	
	\[
	(-1)^{(j_1 + j_2 + \cdots + j_{n-k}) + (j_1 + j_2 + \cdots + j_{n-k})} (-\bm A)_{j_1', j_2', \cdots, j_k'} = (-1)^k \bm A_{j_1', j_2', \cdots, j_k'}
	\]
	其中 $ (j_1', j_2', \cdots, j_k') = (1, 2, \cdots, n) \setminus \{j_1, j_2, \cdots, j_{n-k}\} $,且 $ j_1' < j_2' < \cdots < j_k' $. 因此第 3 种类型的行列式的值为
	\[
	(-1)^k \bm A_{j_1', j_2', \cdots, j_k'} \lambda^{n-k}. 
	\]
	由于 $ 1 \leqslant j_1' < j_2' < \cdots < j_k' \leqslant n $,因此 $ |\lambda\bm{I} - \bm A| $ 中 $ \lambda^{n-k} $ 的系数为
	\[
	(-1)^k \sum_{1 \leqslant j_1' < j_2' < \cdots < j_k' \leqslant n} \bm A_{j_1', j_2', \cdots, j_k'},
	\]
	其中 $ k = 1, 2, \cdots, n-1 $. 特别地,当 $ k = 1 $ 时,得到 $ |\lambda\bm{I} - \bm A| $ 中 $ \lambda^{n-1} $ 的系数为
	\[
	-(a_{11} + a_{22} + \cdots + a_{nn}) = -\text{tr}(\bm A). 
	\]
	因此
	\[
	|\lambda\bm{I} - \bm A| = \lambda^n - \text{tr}(\bm A) \lambda^{n-1} + \cdots + (-1)^k \sum_{1 \leqslant j_1' < j_2' < \cdots < j_k' \leqslant n} \bm A_{j_1', j_2', \cdots, j_k'} \lambda^{n-k} + \cdots + (-1)^n |\bm A|. 
	\]
\end{proof}
\begin{definition}{代数重数和几何重数}{multiplicity}
	设 $\mathcal{A}$ 是 $n$ 维线性空间 $V$ 上的线性变换,$\lambda_0$ 是 $\mathcal{A}$ 的一个特征值,$V_0$ 是属于 $\lambda_0$ 的特征子空间,称 $\dim V_0$ 为 $\lambda_0$ 的\textbf{度数}或\textbf{几何重数}. $\lambda_0$ 作为 $\mathcal{A}$ 的特征多项式根的重数称为 $\lambda_0$ 的重数或代数重数. 
\end{definition}
\begin{proposition}{几何重数小于等于代数重数}{ge leq al}
	设 $\mathcal{A}$ 是 $n$ 维线性空间 $V$ 上的线性变换,$\lambda_0$ 是 $\mathcal{A}$ 的一个特征值,则 $\lambda_0$ 的度数总是小于等于 $\lambda_0$ 的重数. 
\end{proposition}
\begin{proof}
	设特征值 $\lambda_0$ 的重数为 $m$,度数为 $t$,又 $V_0$ 是属于 $\lambda_0$ 的特征子空间,则 $\dim V_0 = t$. 设 $\{\bm{e}_1, \cdots, \bm{e}_t\}$ 是 $V_0$ 的一组基. 由于 $V_0$ 中的非零向量都是 $\mathcal{A}$ 关于 $\lambda_0$ 的特征向量,故
	\[
	\mathcal{A}(\bm{e}_i) = \lambda_0 \bm{e}_i, \quad i = 1, \cdots, t.
	\]
	将 $\{\bm{e}_1, \cdots, \bm{e}_t\}$ 扩充为 $V$ 的一组基,记为 $\{\bm{e}_1, \cdots, \bm{e}_t, \bm{e}_{t+1}, \cdots, \bm{e}_n\}$,则 $\mathcal{A}$ 在这组基下的表示矩阵为
	\[
	\bm A = \begin{pmatrix}
		\lambda_0\bm I_t & * \\
		O & \bm B
	\end{pmatrix},
	\]
	其中 $\bm B$ 是一个 $n-t$ 阶方阵. 因此,线性变换 $\mathcal{A}$ 的特征多项式具有如下形状:
	\[
	|\lambda I_V - \mathcal{A}| = |\lambda \bm I_n - \bm A| = (\lambda - \lambda_0)^t |\lambda\bm I_{n-t} - \bm B|,
	\]
	这表明 $\lambda_0$ 的重数至少为 $t$,即 $t \leqslant m$. 
\end{proof}
\begin{definition}{特征向量系}{}
	设 $\mathcal{A}$ 是 $n$ 维线性空间 $V$ 上的线性变换,若 $\mathcal{A}$ 的任一特征值的度数等于重数,则称 $\mathcal{A}$ 有完全的特征向量系. 
\end{definition}
\begin{tcolorbox}[pikachu]
	\begin{theorem}
		若 $B$ 与 $\bm A$ 相似,则 $B$ 与 $\bm A$ 具有相同的特征多项式,从而具有相同的特征值 (计重数). 
	\end{theorem}
\end{tcolorbox}
\begin{proof}
	设 $\bm B = \bm P^{-1}\bm{AP}$,其中 $\bm P$ 是可逆阵,则
	\[ |\lambda \bm I_n -\bm B| = |\bm P^{-1}(\lambda \bm I_n - \bm A)\bm P| = |\bm P^{-1}||\lambda \bm I_n - \bm A||\bm P| = |\lambda\bm I_n -\bm A|. \]
	因此相似矩阵必有相同的特征多项式,从而必有相同的特征值 (计重数). 
\end{proof}

\subsection{对角化}
\begin{definition}{可对角化线性变换和矩阵}{}
	设$n\in\mathbb Z_{\geq0}$,若$n$维线性空间$V$有基$\{\bm  e_1,\bm e_2,\cdots,\bm  e_n\}$使得每个$\bm e_i$都是$T$的特征向量,则称$T$在$\mathbb F$上是\textbf{可对角化的}.
	
	若将矩阵$\bm A\in M_{n\times n}(\mathbb F)$看作线性映射$\mathbb F^{n}\to\mathbb F^{n}$,则$\bm A$在$\mathbb F$上可对角化相当于存在可逆矩阵$\mathbb P\in M_{n\times n}(\mathbb F)$,使得$\bm T=\bm  P^{-1}\bm{AP}$为对角阵. 
\end{definition}
\begin{tcolorbox}[pikachu]
	\begin{theorem}[可对角化的条件1]
		数域 $  \mathbb K $ 上 $ n $ 级矩阵 $ A $ 可对角化的充分必要条件是 $ A $ 有 $ n $ 个线性无关的特征向量 $ \boldsymbol{a}_1, \boldsymbol{a}_2, \cdots, \boldsymbol{a}_n $. 
	\end{theorem}
\end{tcolorbox}
\begin{proof}
	若 $ n $ 维线性空间 $ V $ 上的线性变换 $ \mathcal{A} $ 在某组基 $ \{\bm e_1, \bm e_2, \cdots, \bm e_n\} $ 下的表示矩阵为对角阵:$\mathrm{diag}\{\lambda_1,\lambda_2,\cdots,\lambda_n\},$ 此时 $ \mathcal{A}(\bm e_i) = \lambda_i\bm e_i $,即 $ \bm e_1, \bm e_2, \cdots, \bm e_n $ 是 $ \mathcal{A} $ 的特征向量,于是 $ \mathcal{A} $ 有 $ n $ 个线性无关的特征向量. 
	
	反过来,若 $ n $ 维线性空间 $ V $ 上的线性变换 $ \mathcal{A} $ 有 $ n $ 个线性无关的特征向量 $\bm e_1, \bm e_2, \cdots, \bm e_n $,则这组向量构成了 $ V $ 的一组基,且 $ \mathcal{A} $ 在这组基下的表示矩阵显然是一个对角阵. 
\end{proof}
\begin{tcolorbox}[pikachu]
	\begin{theorem}\label{thm:eigenvalue-oplus}
		若 $ \lambda_1, \lambda_2, \cdots, \lambda_k $ 为 $ n $ 维线性空间 $ V $ 上的线性变换 $ \mathcal{A} $ 的不同的特征值,则
		\[
		V_1 + V_2 + \cdots + V_k = V_1 \oplus V_2 \oplus \cdots \oplus V_k.
		\]
	\end{theorem}
\end{tcolorbox}
\begin{proof}
	对 $ k $ 用数学归纳法. 若 $ k = 1 $,结论显然成立. 现设对 $ k-1 $ 个不同的特征值 $ \lambda_1, \lambda_2, \cdots, \lambda_{k-1} $,它们相应的特征子空间 $ V_1, V_2, \cdots, V_{k-1} $ 之和是直和. 我们要证明 $ V_1, V_2, \cdots, V_{k-1}, V_k $ 之和为直和,这只需证明:
	\begin{equation}\label{eigenvalue-oplus-2.1} 
		V_k \cap (V_1 + V_2 + \cdots + V_{k-1}) = \{\mathbf 0\}
	\end{equation}
	即可,设 $ \bm{v} \in V_k \cap (V_1 + V_2 + \cdots + V_{k-1}) $,则\begin{equation}\label{eigenvalue-oplus-2.2}
		\bm{v} = \bm{v}_1 + \bm{v}_2 + \cdots + \bm{v}_{k-1},
	\end{equation}
	其中 $ \bm{v}_i \in V_i (i = 1, 2, \cdots, k-1) $. 在 \eqref{eigenvalue-oplus-2.2}式两边作用 $ \mathcal{A} $,得
	\[
	\mathcal{A}(\bm{v}) = \mathcal{A}(\bm{v}_1) + \mathcal{A}(\bm{v}_2) + \cdots + \mathcal{A}(\bm{v}_{k-1}).
	\]
	但 $ \bm{v}, \bm{v}_1, \bm{v}_2, \cdots, \bm{v}_{k-1} $ 都是 $ \mathcal{A} $ 的特征向量或零向量,因此
	\begin{equation}\label{eigenvalue-oplus-2.3}
		\lambda_k \bm{v} = \lambda_1 \bm{v}_1 + \lambda_2 \bm{v}_2 + \cdots + \lambda_{k-1} \bm{v}_{k-1}.
	\end{equation}
	在 \eqref{eigenvalue-oplus-2.2} 式两边乘以 $ \lambda_k $ 减去 \eqref{eigenvalue-oplus-2.3}  式得
	\[
	\{\mathbf 0\} = (\lambda_k - \lambda_1) \bm{v}_1 + (\lambda_k - \lambda_2) \bm{v}_2 + \cdots + (\lambda_k - \lambda_{k-1}) \bm{v}_{k-1}.
	\]
	由归纳假设,$ V_1 + V_2 + \cdots + V_{k-1} $ 是直和,因此 $ (\lambda_k - \lambda_i) \bm{v}_i = 0 $,而 $ \lambda_k - \lambda_i \neq 0 $,从而 $ \bm{v}_i = 0 (i = 1, 2, \cdots, k-1) $. 这就证明了\eqref{eigenvalue-oplus-2.1} 式. 
\end{proof}
\begin{corollary}{}{}
	线性变换 $ \mathcal{A} $ 属于不同特征值的特征向量必线性无关. 
\end{corollary}
\begin{corollary}{}{}
	若 $ n $ 维线性空间 $ V $ 上的线性变换 $ \mathcal{A} $ 有 $ n $ 个不同的特征值,则 $ \mathcal{A} $ 必可对角化. 
\end{corollary}
\begin{corollary}{}{cor-2}
	若线性变换 $ \mathcal{A} $ 的特征多项式没有重根,则 $ \mathcal{A} $ 可对角化. 
\end{corollary}
注意推论\ref{cor:cor-2}只是可对角化的充分条件而非必要条件,比如说纯量变换 $ \mathcal{A} = cI_V $ 当然可对角化,但 $ \mathcal{A} $ 的 $ n $ 个特征值都是 $ c $. 由定理\ref{thm:eigenvalue-oplus},我们还可以得到可对角化的另一个充分必要条件. 
\begin{corollary}{可对角化的充分必要条件}{}
	设 $ \mathcal{A} $ 是 $ n $ 维线性空间 $ V $ 上的线性变换,$ \lambda_1, \lambda_2, \cdots, \lambda_k $ 是 $ \mathcal{A} $ 的全部不同的特征值,$ V_i (i = 1, 2, \cdots, k) $ 是特征值 $ \lambda_i $ 的特征子空间,则 $ \mathcal{A} $ 可对角化的充分必要条件是
	\[
	V = V_1 \oplus V_2 \oplus \cdots \oplus V_k.
	\]
\end{corollary}
\begin{proof}
	先证充分性. 设
	\[
	V = V_1 \oplus V_2 \oplus \cdots \oplus V_k,
	\]
	分别取 $ V_i $ 的一组基 $ \{\bm e_{i1}, \bm e_{i2}, \cdots, \bm e_{it_i}\} (i = 1, 2, \cdots, k) $,则这些向量拼成了 $ V $ 的一组基,并且它们都是 $ \mathcal{A} $ 的特征向量. 因此 $ \mathcal{A} $ 有 $ n $ 个线性无关的特征向量,从而 $ \mathcal{A} $ 可对角化. 
	
	再证必要性. 设 $ \mathcal{A} $ 可对角化,则 $ \mathcal{A} $ 有 $ n $ 个线性无关的特征向量 $ \{\bm e_1, \bm e_2, \cdots, \bm e_n\} $,它们构成了 $ V $ 的一组基. 不失一般性,可设这组基中前 $ t_1 $ 个是关于特征值 $ \lambda_1 $ 的特征向量;接下去的 $ t_2 $ 个是关于特征值 $ \lambda_2 $ 的特征向量;……;最后 $ t_k $ 个是关于特征值 $ \lambda_k $ 的特征向量. 对任一 $ \alpha \in V $,设 $ \bm\alpha = a_1 \bm e_1 + a_2 \bm e_2 + \cdots + a_n \bm e_n $,则 $\bm\alpha $ 可写成 $ V_1, V_2, \cdots, V_k $ 中向量之和,因此由定理\ref{thm:eigenvalue-oplus}可得
	\[
	V = V_1 + V_2 + \cdots + V_k = V_1 \oplus V_2 \oplus \cdots \oplus V_k. \qedhere
	\]
\end{proof}
\begin{tcolorbox}[pikachu]
	\begin{theorem}[可对角化的充分必要条件]
		设 $ \mathcal A $ 是 $ n $ 维线性空间 $ V $ 上的线性变换,则 $ \mathcal A $ 可对角化的充分必要条件是 $ \mathcal A $ 有完全的特征向量系,即几何重数等于代数重数. 
	\end{theorem}
\end{tcolorbox}
\begin{proof}
	设 $ \lambda_1, \lambda_2, \cdots, \lambda_k $ 是 $ \mathcal A $ 的全部不同的特征值,它们对应的特征子空间、重数和度数分别记为 $ V_i, m_i, t_i (i = 1, 2, \cdots, k) $. 由重数的定义\ref{def:multiplicity}以及命题\ref{pro:ge leq al}可知
	\[
	m_1 + m_2 + \cdots + m_k = n, \quad t_i \leqslant m_i, \quad i = 1, 2, \cdots, k.
	\]
	由推论\ref{cor:cor-2},我们只要证明 $ \mathcal A $ 有完全的特征向量系当且仅当 $ V = V_1 \oplus V_2 \oplus \cdots \oplus V_k $. 若 $ V = V_1 \oplus V_2 \oplus \cdots \oplus V_k $,则
	\[
	n = \dim V = \dim(V_1 \oplus V_2 \oplus \cdots \oplus V_k) = \dim V_1 + \dim V_2 + \cdots + \dim V_k = \sum_{i=1}^k t_i \leqslant \sum_{i=1}^k m_i = n,
	\]
	因此 $ t_i = m_i (i = 1, 2, \cdots, k) $,即 $ \mathcal A $ 有完全的特征向量系. 反过来,若 $ \mathcal A $ 有完全的特征向量系,则	
	\[
	\dim(V_1 \oplus V_2 \oplus \cdots \oplus V_k) = \sum_{i=1}^k t_i = \sum_{i=1}^k m_i = n = \dim V,
	\]
	从而 $ V = V_1 \oplus V_2 \oplus \cdots \oplus V_k $ 成立. 
\end{proof}
\begin{corollary}{}{}
	$ n $ 维线性空间 $ V $ 上的线性变换 $ \mathcal{A} $ 可对角化当且仅当 $ \mathcal{A} $ 的属于不同特征值的特征子空间的维数之和等于 $ n $. 
\end{corollary}

\section{环、域与多项式}
\subsection{环与域}
非空集 $S$ 上的 $n$ 元运算($n \in \mathbb{Z}_{\geqslant 1}$)无非是指一个映射 $S^n \rightarrow S$;譬如加法 + 和乘法 $\cdot$ 都是 $\mathbb{Z}$ 上的二元运算. 对于一般的二元运算
\[
\star : S \times S \rightarrow S;
\]
习惯的做法是将 $\star(s_1, s_2)$ 写成 $s_1 \star s_2$. 对于可以理解为某种乘法的运算,通常以 $\cdot$ 标记;简写 $s_1 s_2 = s_1 \cdot s_2$ 也是常用的. 
\begin{definition}{环}{ring}
	设 $R$ 是非空集合,在$R$上定义了二元运算:$+ : R \times R \rightarrow R$ 和 $\cdot : R \times R \rightarrow R$ ,对任意的$x,y,z\in R$,使得以下条件成立:
	\begin{enumerate}
		\item 加法运算满足以下条件:
		\begin{enumerate}[(1)]
			\item 结合律: $(x + y) + z = x + (y + z)$;
			\item 零元性质 :$x + 0_R = x = 0_R + x$.
			\item 交换律: $x + y = y + x$.
			\item 加法逆元: 对所有 $x$ 皆存在 $-x$ 使得 $x + (-x) = 0_R$.
		\end{enumerate}
		\item 乘法运算 $x \cdot y$ 也简写为 $xy$,它满足以下条件:
		\begin{enumerate}[(1)]
			\item 结合律:$(xy)z = x(yz)$;
			\item 幺元性质\footnote{不同的教材对环的定义不同体现在是否含有乘法幺元,例如参考书\cite{5}中定义不含乘法幺元(即对乘法构成半群),而参考书\cite{6}中定义含有乘法幺元. 含有乘法幺元的环具有更多好的性质,因此本笔记的环均指\textbf{含幺环}. }:$x \cdot 1_R = x = 1_R \cdot x$;
		\end{enumerate}
		\item 乘法对加法满足
			\begin{itemize}
				\item 分配律: $(x + y)z = xz + yz, \quad z(x + y) = zx + zy$.
			\end{itemize}
	\end{enumerate}
	则称$(R, +, \cdot, 0_R, 1_R)$为一个\textbf{环}. 不致混淆时,我们也把 $0_R, 1_R$ 简记为 $0, 1$,并以 $R$ 代表 $(R, +, \cdot, 0_R, 1_R)$. 为了方便,我们也将 $x + (-y)$ 写作 $x - y$. 
\end{definition}
由环的定义不难推出如下性质:
\begin{enumerate}
	\item 结合律确保任意有限多个元素的加法和乘法可以不带括号地写作 $x + y + z, xyz$ 等.
	\item 分配律具有双边的版本:
	\[
	a(x + y)b = (ax + ay)b = axb + ayb.
	\]
	\item 加法和乘法幺元\footnote{也就是单位元,加法幺元又称零元. }都由各自的幺元性质唯一确定. 
	\begin{proof}
		设 $0_R$ 和 $0'_R$ 皆满足加法幺元性质, $1_R$ 和 $1'_R$ 皆满足乘法幺元性质, 则
		\[
		0_R = 0_R + 0'_R = 0'_R, \quad 1_R = 1_R \cdot 1'_R = 1'_R.\qedhere
		\]
	\end{proof}
	\item 加法满足消去律: 若 $x + y = x' + y$, 等式两边同加 $-y$, 应用加法结合律得 $x = x + y + (-y) = x' + y + (-y) = x'$.
	\item 任何 $x$ 的加法逆元 $-x$ 皆唯一, 这是因为若 $x + x' = 0 = x + x''$, 则加法消去律蕴涵 $x' = x''$. 因此取加法逆元 $x \mapsto -x$ 也可以视为 $R$ 上的一元运算.
	\item 从加法逆元的唯一性和 $x + (-x) = 0 = (-x) + x$ 立见 $-(-x) = x$.
	\item 恒等式 $x \cdot 0 = 0 = 0 \cdot x$ 成立. 以第一个等号为例, 我们有$x \cdot 0 = x \cdot (0 + 0) = x \cdot 0 + x \cdot 0$, 对两端应用消去律可得 $x \cdot 0 = 0$.
	\item 恒等式 $(-x)y = -xy = x(-y)$ 成立, 这是因为
	\[
	(-x)y + xy = (-x + x)y = 0 \cdot y = 0, \quad x(-y) + xy = x(-y + y) = x \cdot 0 = 0
	\]
	和加法逆元的唯一性.
	\item 作为上式的应用, 我们有 $(-1) \cdot y = -y$ 和 $-x = x \cdot (-1)$; 特别地, 代入 $x = -1$ 给出 $(-1) \cdot (-1) = 1$.
\end{enumerate}
\begin{remark}
	最平凡的环是零环: 这是只有单个元素 $1 = 0$ 的环. 另一方面, 非零环必然满足 $1 \neq 0$, 否则任何 $x$ 都满足 $x = x \cdot 1 = x \cdot 0 = 0$.
\end{remark}
\begin{example}{Gauss整数环}{}
	我们经常遇到的很多数的集合,在数的普通加法和乘法下都构成环. 例如,任何数域都是环. 除此之外,很多本身不是域的数的集合也构成环. 例如,全体整数的集合 $\mathbb{Z}$ 在加法和乘法下也构成环. 现设 $m \in \mathbb{Z}$,令
	\[
	\mathbb{Z}[\sqrt{m}] = \{ a + b\sqrt{m} \mid a, b \in \mathbb{Z} \}.
	\]
	则 $\mathbb{Z}[\sqrt{m}]$ 也构成环. 特别地,当 $m = -1$ 时有
	\[
	\mathbb{Z}[\sqrt{-1}] = \{ a + b\sqrt{-1} \mid a, b \in \mathbb{Z} \}.
	\]
	这是历史上非常著名的环的例子,称为 \textbf{Gauss 整数环}. 
\end{example}
\begin{example}{多项式环与矩阵环}{}
	项式的集合和矩阵的集合都构成环. 具体来说,设 $\mathbb{P}$ 为一个数域,令 $\mathbb{P}[x]$ 为 $\mathbb{P}$ 上全体以 $x$ 为文字的一元多项式的集合,则 $\mathbb{P}[x]$ 在多项式的加法和乘法下构成环,称为数域 $\mathbb{P}$ 上的 \textbf{一元多项式环},或简称为 $\mathbb{P}$ 上的多项式环. 类似地,记 $\mathbb{P}^{n \times n}$ 为 $\mathbb{P}$ 上全体矩阵构成的集合,则 $\mathbb{P}^{n \times n}$ 在矩阵的加法和乘法下构成环,称为 $\mathbb{P}$ 上的 $n$ 阶方阵环. 
\end{example}
\begin{example}{某些函数构成的环}{}
	记实数轴上全体连续函数构成的集合为 $C(\mathbb{R})$,定义加法与乘法为
	\[
	(f + g)(x) = f(x) + g(x),
	\]
	\[
	(fg)(x) = f(x)g(x), \quad x \in \mathbb{R}, \, f, \, g \in C(\mathbb{R}),
	\]
	则容易验证 $C(\mathbb{R})$ 构成环. 同样地,记 $\mathbb{R}$ 上全体光滑函数(即具有任何阶的连续导数)的集合为 $C^{\infty}(\mathbb{R})$,则在上述两种运算下 $C^{\infty}(\mathbb{R})$ 构成环. 
	
	上述环有多种形式的推广. 例如,对任何闭区间 $[a, b] \subset \mathbb{R}$,设 $C([a, b])$ 为 $[a, b]$ 上全体连续函数构成的集合,则 $C([a, b])$ 在上述两种运算下构成环. 如果考虑多元函数,则欧几里得空间 $\mathbb{R}^n$ 上全体光滑函数的集合 $C^{\infty}(\mathbb{R}^n)$ 在上述加法和乘法下构成环. 对 $C^{\infty}(\mathbb{R}^n)$ 的研究在微分几何中具有重要意义. 
\end{example}
	对于任意 $n \in \mathbb{Z}_{\geqslant 0}$ 和 $r \in R$, 我们引入自明的写法\begin{equation}
		\begin{aligned}
			n \cdot r = nr := \underbrace{r + \cdots + r}_{n \text{ 项}}, \quad n \geqslant 1,\\0 \cdot r := 0, \quad (-n) \cdot r = (-(n \cdot r)) := -(n \cdot r)
		\end{aligned}
	\end{equation}
容易验证\begin{equation}
	\begin{aligned}
		n(r + r') = nr + nr', \quad (n + m)r = nr + mr,\\
		(nm)r = n(mr), \quad (nr)r' = n(rr'),\\
		r(n \cdot 1_R) = nr = (n \cdot 1_R)r\\
	\end{aligned}
\end{equation}
对所有 $n, m \in \mathbb{Z}$ 和 $r, r' \in R$ 皆成立. 

对于带有二元运算 $\star$ 的非空集 $S$ 及其子集 $S'$, 如果对所有 $s_1, s_2 \in S'$ 都有 $s_1 \star s_2 \in S'$, 则我们顺理成章地说 $S'$ 对运算 $\star$ \textbf{封闭}, 对于一般的 $n$ 元运算当然也有类似的说法. 封闭性可以用来定义代数结构的子结构, 以下仍以环为例.
\begin{definition}{子环}{subring}
	如果 $R$ 的子集 $R_0$ 包含 $0_R, 1_R$, 而且在加法, 乘法运算和加法取逆 $x \mapsto -x$ 之下封闭, 则 $(R_0, +, \cdot, 0_R, 1_R)$ 也是环, 称为 $R$ 的\textbf{子环}.
\end{definition}

\begin{example}{环的中心}{center of ring}
	环 $R$ 的中心定义为
	\[
	Z(R) := \{z \in R : \forall x \in R, zx = xz\}.
	\]
	容易看出 $Z(R)$ 是 $R$ 的子环.
\end{example}
\begin{definition}{逆}{inverse}
	设 $x$ 是环 $R$ 的元素. 若存在 $y \in R$ 使得 $xy = 1$ (或 $yx = 1$), 则称 $y$ 为 $x$ 的\textbf{右逆} (或\textbf{左逆}), 而 $x$ 右可逆 (或左可逆). 若 $x$ 左右皆可逆, 则称 $x$ \textbf{可逆}. 由 $R$ 的可逆元构成的子集记为 $R^\times$.
\end{definition}
我们可以证明:
\begin{lemma}{}{}
	如果环 $R$ 的元素 $x$ 可逆, 则 $x$ 的左逆也必然是右逆, 而且存在唯一的 $x^{-1} \in R$ 使得 $x^{-1}x = 1 = xx^{-1}$; 此时 $(x^{-1})^{-1} = x$.
\end{lemma}
\begin{proof}
	设 $x$ 可逆, $x$ 为其左逆而 $x_R$ 为其右逆. 由乘法结合律有
	\[
	x_R = 1 \cdot x_R = (x_L \cdot x) \cdot x_R = x_L \cdot (x \cdot x_R) = x_L \cdot 1 = x_L.
	\]
	这就说明左逆等于右逆, 反之亦然. 
	
	另一方面,如果 $x_L$ 和 $x'_L$ 都是 $x$ 的左逆, $x_R$ 和 $x'_R$ 都是 $x$ 的右逆, 则将左逆和右逆的四种组合代入上式, 可得
	\[
	x_L = x_R, \quad x_L = x'_R, \quad x'_L = x_R, \quad x'_L = x'_R;
	\]
	特别地, $x_L = x'_L$ 而 $x_R = x'_R$. 综上, 在$x$可逆的前提下, 左逆等于有右逆,并且唯一, 可以合理地记为 $x^{-1}$.
\end{proof}
注意到 $R^\times$ 包含 1 (显然 $1^{-1} = 1$), 而且对乘法运算封闭: 从 $y^{-1}x^{-1}xy = 1 = xyy^{-1}x^{-1}$ 可得
\[
(xy)^{-1} = y^{-1}x^{-1}, \quad x, y \in R^\times.
\]
进一步, 性质 $(x^{-1})^{-1} = x$ 说明 $R^\times$ 对取逆运算 $x \mapsto x^{-1}$ 也封闭.
对于环中的元素 $r \in R$ 及其 $n \in \mathbb{Z}_{\geqslant 1}$, 我们记
\[
r^n = \underbrace{r \cdots r}_{n \text{ 项}};
\]
此外 $r^0 := 1$. 若 $r \in R^\times$, 则进一步记
\[
r^{-n} := (r^n)^{-1} = (r^{-1})^n, \quad n \in \mathbb{Z}_{\geqslant 1}.
\]
我们总有等式 $r^{m+n} = r^m r^n$; 当 $r$ 可逆时, 此式对 $m$ 或 $n$ 为负的情形同样成立. 同理, $r^{mn} = (r^m)^n$.
\begin{definition}{交换环}
	如果环 $R$ 的乘法满足交换律 $xy = yx$, 则称 $R$ 为\textbf{交换环}.
\end{definition}
因此 $R$ 是交换环当且仅当 $Z(R) = R$.
\begin{definition}{交换环与域}{communicative ring}
	满足 $R^\times = R \setminus \{0\}$ (换言之: 零不可逆, 而非零元皆可逆) 的环称为\textbf{除环}. 交换除环称为\textbf{域}. 域的子环如果也构成域, 则称之为\textbf{子域}.
\end{definition}
由于域的乘法顺序可换, 在域中可以合理地将 $xy^{-1}$ 写作 $x/y$ 或 $\dfrac{x}{y}$, 前提是 $y \neq 0$.
\begin{example}{}{}
	对于寻常的乘法和加法运算, $\mathbb{C}$ 是域, 而 $\mathbb{R}, \mathbb{Q}$ 都是 $\mathbb{C}$ 的子域, 而子环 $\mathbb{Z}$ 不是域; 事实上 $\mathbb{Z}^\times = \{\pm 1\}$.
\end{example}
\begin{definition}{整环}{integral ring}
	非零交换环 $R$ 若满足 $x, y \neq 0 \implies xy \neq 0$, 则称为\textbf{整环}.
\end{definition}
整环的子环显然也是整环. 在整环中乘法对所有非零元都有消去律, 这是因为 $x \neq 0$ 和 $xy = xz$ 蕴涵 $x(y-z) = 0$, 因而蕴涵 $y = z$. 域自动是整环, 这是因为 $x \neq 0$ 和 $xy = 0$ 给出 $y = x^{-1}xy = x^{-1} \cdot 0 = 0$.
\begin{example}{同余类构成的环}{}
	设 $n \in \mathbb{Z}$. 选定 $n \in \mathbb{Z}$, 记 $\mathbb{Z}$ 对等价关系 $\text{mod } n$ 的商集\footnote{请看笔者上一篇文章高等代数笔记3:线性空间->线性映射 - 晨锦辉永生之语的文章 - 知乎\url{https://zhuanlan.zhihu.com/p/1890514261077381838}}为 $\mathbb{Z}/n\mathbb{Z}$, 或简记为 $\mathbb{Z}n$; 其中的等价类也称为 $\text{mod } n$ \textbf{同余类}. 在 $\mathbb{Z}/n\mathbb{Z}$ 上定义加法和乘法运算如下
	\[
	[x][y] := [xy], \quad [x] + [y] := [x + y],
	\]其中 $x, y \in \mathbb{Z}$. 运算是良定义的, 也就是说运算产物仅依赖同余类 $[x]$ 和 $[y]$ 而不是 $x$ 和 $y$ 的具体取法, 这是容易由初等数论的知识证明的. 取 $0_{\mathbb{Z}/n\mathbb{Z}} := [0], 1_{\mathbb{Z}/n\mathbb{Z}} := [1]$, 立见 $\mathbb{Z}/n\mathbb{Z}$ 对此运算成为交换环. 注意到 $\mathbb{Z}/0\mathbb{Z} = \mathbb{Z}$ 而 $\mathbb{Z}/(-n)\mathbb{Z} = \mathbb{Z}/n\mathbb{Z}$, 因此以下不妨设 $n \in \mathbb{Z}_{\geqslant 1}$, 此时 $\mathbb{Z}/n\mathbb{Z}$ 恰有 $n$ 个元素; 它是零环当且仅当 $n = 1$.
\end{example}
注意到 $[x] \in (\mathbb{Z}/n\mathbb{Z})^\times$ 相当于说同余式 $xy \equiv 1 \pmod{n}$ 有解 $y \in \mathbb{Z}$. 根据Bézout定理, 此式有解等价于 $x$ 和 $n$ 互素; 换言之,
\[
(\mathbb{Z}/n\mathbb{Z})^\times = \{[x] : x \in \mathbb{Z}, x, n \text{ 互素}\};
\]
基于 Euler 函数 $\mathcal A$ 的定义\footnote{$\mathcal A(n)$定义为不超过$n$而与$n$互素的正整数个数,也即正是与n 互素的 mod n 同余类个数.}, 由此就得出 $|(\mathbb{Z}/n\mathbb{Z})^\times| = \mathcal A(n)$. 作为推论,
\[
\mathbb{Z}/n\mathbb{Z} \text{ 为域} \iff \mathcal A(n) = n - 1 \iff n \text{ 为素数}.
\]
我们也容易证明,$\mathbb{Z}/n\mathbb{Z}$ 为整环当且仅当它是域.

设 $p$ 为素数. 域 $\mathbb{Z}/p\mathbb{Z}$ 是有限域的初步例子. 鉴于它的重要性, 我们另外引入符号
\[
\mathbb{F}_p := \mathbb{Z}/p\mathbb{Z}.
\]
\begin{definition}{环的直积}{}
	取一族环 $(R_i)_{i \in I}$, 下标 $i$ 遍历某个非空集 $I$. 下面使用某种途径从已有的环构造新环,称作 $(R_i)_{i \in I}$ 的\textbf{直积}.  
	\begin{enumerate}[(1)]
		\item 在 $\prod\limits_{i \in I} R_i$ 上逐分量地定义加法和乘法, 分别写作
		\[
		\underbrace{(r_i)_i + (r'_i)_i := (r_i + r'_i)_{i \in I}}_{\prod_i R_i\text{的加法}}, \quad \underbrace{(r_i)_i \cdot (r'_i)_i := (r_i \cdot r'_i)_{i \in I}}_{\prod_i R_i\text{的乘法}}.
		\]
		\item 定义零元 0 为 $(0_i)_i$, 幺元 1 为 $(1_i)_i$, 下标 $i$ 代表它们分别是 $R_i$ 中的零元和幺元. 
	\end{enumerate}
	这样我们就可以在每个 $R_i$ 上来检验环的定义\ref{def:ring}, 就以加法结合律为例:
	\[
	((r_i)_i + (r''_i)_i) + (r'_i)_i = ((r_i + r''_i) + r'_i)_i = (r_i + (r'_i + r''_i))_i = (r_i)_i + (r'_i)_i + (r''_i)_i,
	\]
	其他情形也是类似的. 容易看出 $-(r_i)_i = (-r_i)_i$. 若 $I = \{1, \ldots, n\}$, 对应的直积也写作 $R_1 \times \cdots \times R_n$ 的形式.  
\end{definition}
接着考虑每个 $R_i$ 都是同一个环 $R$ 的特例, 这时 $\prod\limits_{i \in I} R_i$ 化为映射集 $R^I = \{f : I \rightarrow R\}$ 相对于逐点或逐元素的运算
\[
(f + g)(i) := f(i) + g(i), \quad (fg)(i) := f(i)g(i), \quad i \in I
\]
所成的环, 方式是让 $f$ 对应 $(f(i))_{i \in I} \in \prod\limits_{i \in I} R$; 特别地, $0_{R^I}$ 是常值映射 $i \mapsto 0_R$, 而 $1_{R^I}$ 是常值映射 $i \mapsto 1_R$. 
\subsection{同态与同构}
\begin{definition}{环同态}{ring homomorphism}
	设 $f : R \rightarrow R'$ 为环之间的映射. 当以下条件成立时, 称 $f$ 为\textbf{环同态}:
	\begin{enumerate}
		\item $f(x + y) = f(x) + f(y)$,
		\item $f(xy) = f(x)f(y)$,
		\item $f(1_R) = 1_{R'}$,
	\end{enumerate}
	其中 $x, y$ 取遍 $R$ 的元素. 从环 $R$映到其自身的同态也称为 $R$ 的\textbf{自同态}.
\end{definition}
由定义不难推出环同态的一些性质:\begin{enumerate}
		\item \textbf{保持零元:}$f(0_R) = 0_{R'}.$
		\begin{proof}
			从 $f(0_R) = f(0_R + 0_R) = f(0_R) + f(0_R)$, 配合 $R'$ 中的加法消去律, 即得 $f(0_R) = 0_{R'}$.
		\end{proof} 
		\item \textbf{保持加法逆元:} $f(-x) = -f(x).$ 这是 $0_{R'} = f(0_R) = f(x + (-x)) = f(x) + f(-x)$ 的推论.
		\item \textbf{保持乘法逆元:} 若 $x \in R^\times$, 则 $f(x) \in (R')^\times$ 而 $f(x^{-1}) = f(x)^{-1}$, 这是因为 $1_{R'} = f(1_R) = f(xx^{-1}) = f(x)f(x^{-1})$.
		\item \textbf{恒等自同态:} 任何环 $R$ 到它自身的恒等映射 $\text{id}_R$ 自动是环同态, 这是环同态的平凡例子.
		\item \textbf{同态的合成:} 若 $f : R \rightarrow R'$ 和 $g : R' \rightarrow R''$ 为环同态, 则 $gf : R \rightarrow R''$ 也是环同态. 这是因为
		\begin{align*}
			&gf(x + y) = g(f(x) + f(y)) = gf(x) + gf(y), \\
			&gf(xy) = g(f(x)f(y)) = gf(x)gf(y), \\
			&gf(1_R) = g(1_{R'}) = 1_{R''}.
		\end{align*}
		\item \textbf{像与子环:}对于环同态 $f : R \rightarrow R'$, 它的像 $f(R)$ 自然是 $R'$ 的子环; 反过来说, 给定环 $R'$ 及其子环 $R \subset R'$, 取 $\iota : R \rightarrow R'$ 为包含映射, 映 $r \in R$ 为 $r$, 则 $\iota$ 自然是环同态.
\end{enumerate}
\begin{example}{同余类上的环同态}{ring homo pmod}
	设$n,m\in\mathbb Z$满足$n\mid m$,考虑映射\begin{align*}
		p_n^m:\mathbb Z/m\mathbb Z\to \mathbb Z/n\mathbb Z\\ [x]_m\mapsto [x]_n.
	\end{align*}
	首先这是良定义的,对于任意$x,y\in\mathbb Z$,成立\[x\equiv y\pmod m\iff m\mid x-y\implies n\mid x-y\iff x\equiv y\pmod n.\]
	根据同余类中加法和乘法的运算法则,我们有\[p_m^n([x]_m+[y]_m)=p_m^n([x+y]_m)=[x+y]_n=[x]_n+[y]_n=p_m^n([x]_m)+p_m^n([y]_m).\]
	同理容易验证\[p_m^n([x]_m[y]_m)=p_m^n([x]_m)p_m^n([y]_m),\]
	这表明$p_m^n$是环同态. 
\end{example}
\begin{definition}{环同构}{ring isomorphism}
	设 $f : R \rightarrow R'$ 为环同态. 如果存在环同态 $g : R' \rightarrow R$ 使得 $gf = \text{id}_R$ 而 $fg = \text{id}_{R'}$, 则称 $f$ 为\textbf{环同构}, 而 $g$ 为 $f$ 的逆. 此时我们也说 $R$ 和 $R'$ \textbf{同构}.
	
	可以用符号 $f : R \xrightarrow{\sim} R'$ 代表映射 $f : R \rightarrow R'$ 是环同构; 在不必指明 $f$ 的场合, 我们也以符号 $R \simeq R'$ 代表环 $R$ 和 $R'$ 同构.
\end{definition}
条件 $gf = \text{id}_R$ 和 $fg = \text{id}_{R'}$ 表明 $f$ 的逆无非是 $f$ 作为映射的逆 $g = f^{-1}$. 反过来说, 容易证环同态 $f$ 如果作为映射是双射, 那么它也是环同构.
\begin{proposition}{同态+双射=同构}{}
	设 $f : R \rightarrow R'$ 为环同态. 如果 $f$ 是集合之间的双射, 则 $f$ 是环同构.
\end{proposition}
\begin{proof}
	问题归结为证 $f$ 的逆映射 $f^{-1}$ 也是环同态. 对 $f(1_R) = 1_{R'}$ 两边取 $f^{-1}$ 可得 $1_R = f^{-1}(1_{R'})$. 对 $f(x + y) = f(x) + f(y)$ 两边取 $f^{-1}$, 并且记 $u = f(x)$, $v = f(y)$, 可得 $f^{-1}(u) + f^{-1}(v) = f^{-1}(u + v)$. 同理可见 $f^{-1}(uv) = f^{-1}(u)f^{-1}(v)$. 由于所有 $u, v \in R'$ 都能表作 $u = f(x)$ 和 $v = f(y)$ 的形式, 综上可见 $f^{-1}$ 确实是环同态.
\end{proof}
恒等映射 $\text{id}_R$ 是同构最简单的例子. 此外, 两个同构 $f$ 和 $g$ 的合成 $gf$ 依然是同构, 以 $f^{-1}g^{-1}$ 为逆.

同构 $f : R \simeq R'$ 不但为集合 $R$ 和 $R'$ 建立了双射, 而且对应元素之间的一切环论运算 (加法, 乘法) 和幺元也在 $f$ 之下相配对. 凡是以环论语言表达的一切性质, 对于同构的环 $R$ 和 $R'$ 都是等价的. 这是代数学中的一条基本原理.
\begin{proposition}{}{ring homo injection}
	设 $\mathbb F$ 为域, $R$ 为非零环, 而 $\mathcal A : F \rightarrow R$ 为环同态. 证明: $\mathcal A$ 为单射.
\end{proposition}
\begin{proof}
	我们有 $\mathcal A(x) = \mathcal A(y) \iff \mathcal A(x - y) = 0$, 所以问题化为证 $x \neq 0 \implies \mathcal A(x) \neq 0$. 但是域 $F$ 中的任意非零元都是可逆的, 而同态映可逆元为可逆元.
\end{proof}

\begin{tcolorbox}[pikachu]
	\begin{theorem}[中国剩余定理 — 同构版本]
		设 $N = n_1 \cdots n_k$, 其中 $n_1, \ldots, n_k \in \mathbb{Z}_{\geqslant 1}$ 两两互素, 则有环同构
		\begin{align*}
			\mathcal A : \mathbb{Z}/N\mathbb{Z} \xrightarrow{\sim} \prod_{i=1}^k \mathbb{Z}/n_i\mathbb{Z}\\ [x]_N\longmapsto([x]_{n_i})^k_{i=1}.
		\end{align*}
	\end{theorem}
\end{tcolorbox}
\begin{proof}
	例\ref{ex:ring homo pmod}业已说明 $[x]_N \mapsto [x]_{n_i}$ 对所有 $i$ 都给出同态 $\mathbb{Z}/N\mathbb{Z} \rightarrow \mathbb{Z}/n_i\mathbb{Z}$. 既然直积的环结构是逐分量定义的, $\mathcal A$ 必保持环结构, 从而是同态.
	
	此外, 映射两端作为集合都有 $N$ 个元素, 基于抽屉原理, 证 $\mathcal A$ 是单射即可. 互素条件在此派上用场: 设 $x, y \in \mathbb{Z}$ 满足 $\mathcal A([x]_N) = \mathcal A([y]_N)$, 则对所有下标 $i$ 都有
	\[
	n_i \mid x - y.
	\]
	既然 $n_1, \ldots, n_k$ 两两互素, 故 $N \mid x - y$, 亦即 $[x]_N = [y]_N$. 单性得证. 
\end{proof}
\subsection{分式域}
设 $ R $ 为整环. 我们考虑集合
\[ \text{Ratio}(R) := \{(f, g) \in R^2 : g \neq 0\}. \]
在 $\text{Ratio}(R)$ 上定义二元关系
\[ (f_1, g_1) \sim (f_2, g_2) \iff f_1 g_2 = f_2 g_1. \]
这是一个等价关系. 反身性和对称性是显然的,只需简单验证传递性:设 $ (f_1, g_1) \sim (f_2, g_2) $ 而 $ (f_2, g_2) \sim (f_3, g_3) $,则 $ R $ 的交换性导致
\[ (f_1 g_2) g_3 = (f_2 g_1) g_3 = (f_2 g_3) g_1 = (f_3 g_2) g_1. \]
因为 $ R $ 是整环,两边消去非零元 $ g_2 $ 便得到 $ f_1 g_3 = f_3 g_1 $,亦即 $ (f_1, g_1) \sim (f_3, g_3) $. 

定义商集 $ \text{Frac}(R) := \text{Ratio}(R) / \sim $. 接着来赋予 $ \text{Frac}(R) $ 环结构. 
\begin{definition}{\text{Frac}(R)的环结构}{}
	\begin{enumerate}[(1)]
		\item \textbf{加法和乘法.} 规定加法和乘法运算:\begin{gather}
			\frac{f_1}{g_1} + \frac{f_2}{g_2} := \frac{f_1 g_2 + g_1 f_2}{g_1 g_2}, \label{fracfield-1}\\\frac{f_1}{g_1} \cdot \frac{f_2}{g_2} := \frac{f_1 f_2}{g_1 g_2}. \label{fracfield-2}
		\end{gather}
		不难验证,\eqref{fracfield-1}式和 \eqref{fracfield-2} 式不依赖于等价类中代表的选择(即良定义的). 以\eqref{fracfield-1}式为例. 设 $\frac{f_1}{g_1} = \frac{f_1'}{g_1'}, \frac{f_2}{g_2} = \frac{f_2'}{g_2'}$,则
		\[
		f_1 g_1' = g_1 f_1', \quad f_2 g_2' = g_2 f_2',
		\]
		于是有\begin{gather}
			f_1 g_1' (g_2 g_2') = g_1 f_1' (g_2 g_2'),\label{fracfield-3}\\f_2 g_2' (g_1 g_1') = g_2 f_2' (g_1 g_1').\label{fracfield-4}
		\end{gather}
		\eqref{fracfield-3}式与 \eqref{fracfield-4} 式相加,得
		\[
		f_1 g_1' g_2 g_2' + f_2 g_2' g_1 g_1' = g_1 f_1' g_2 g_2' + g_2 f_2' g_1 g_1',
		\]
		由此得出
		\[
		\frac{f_1 g_2 + g_1 f_2}{g_1 g_2} = \frac{f_1' g_2' + g_1' f_2'}{g_1' g_2'},
		\]
		即\begin{equation}
			\frac{f_1}{g_1} + \frac{f_2}{g_2} = \frac{f_1'}{g_1'} + \frac{f_2'}{g_2'}.
		\end{equation}
		类似地可以证明,用 \eqref{fracfield-2} 式规定 $ R $ 中的乘法运算是合理的. 容易验证,上述定义的加法和乘法都满足交换律、结合律,并且满足分配律. 
		\item \textbf{零元.} $ \frac{0}{1} $ 是 $ \text{Frac}(R) $ 中的零元,记作 $ 0 $;根据环的性质,可以定义$ \frac{f}{g} $ 的负元 $ \frac{-f}{g}=-\dfrac fg $. 
		\item \textbf{乘法幺元.} $ \frac{1}{1} $ 是 $ \text{Frac}(R) $ 的单位元,记作 1. 
	\end{enumerate}
\end{definition}
不难验证,在上述定义下,$\text{Frac}(R)$成为\textbf{交换环}\footnote{当然含幺.}. 

事实上,由定义\ref{def:communicative ring},$\text{Frac}(R)$构成一个域,称为整环$R$的\textbf{分式域}. 
\begin{proposition}{分式域}{fracfield}
	交换(除)环$\text{Frac}(R)$的非零元皆可逆. 
\end{proposition}
\begin{proof}
	对于 $ \text{Frac}(R) $ 中每一个非零元 $ \frac{f}{g} $,都存在 $ \frac{g}{f} \in \text{Frac}(R) $,使得
	\[
	\frac{f}{g} \cdot \frac{g}{f} = \frac{fg}{gf} = \frac{1}{1} = 1, \quad \frac{g}{f} \cdot \frac{f}{g} = \frac{gf}{fg} = \frac{1}{1} = 1,
	\]
	这表明 $ \frac{f}{g} $ 是可逆的,$ \frac{g}{f} $ 是 $ \frac{f}{g} $ 的逆元,记作 $ \left( \frac{f}{g} \right)^{-1} $,即
	\[
	\left( \frac{f}{g} \right)^{-1} := \frac{g}{f}.
	\]
	由于 $ \text{Frac}(R) $ 的每个非零元都可逆,因此可以在 $ \text{Frac}(R) $ 中定义除法如下:
	
	设 $ \frac{f_2}{g_2} \neq 0 $,对于任意 $ \frac{f_1}{g_1} \in \text{Frac}(R) $,规定
	\[
	\frac{f_1}{g_1} \bigg/ \frac{f_2}{g_2} := \frac{f_1}{g_1} \cdot \left( \frac{f_2}{g_2} \right)^{-1}.
	\]
	
	再将$ \text{Frac}(R) $ 中的减法运算的定义取环中的减法定义即可. 
\end{proof}
\begin{remark}
	此时映射 $f \mapsto [f, 1]$ 将 $R$ 自然地嵌入为 $ \text{Frac}(R)$ 的子环. 
\end{remark}
分式的基本性质现在可以证明如下:
设 $\frac{f}{g} \in R$. 任取 $h(x) \in R\setminus\{0\}$,由于 $ fgh = gfh $,因此\begin{equation}\label{fracfield-5}
	\frac{f}{g} = \frac{fh}{gh},
\end{equation}
将 \eqref{fracfield-5} 式从右到左看,即得到:分子与分母可以消去同一个非零公因式. 
\begin{lemma}{}{}
	对于一个非零的分式 $\frac{f}{g}$,分子的次数减去分母的次数所得的差 $\deg f - \deg g$ 不依赖于等价类的代表的选取. 
\end{lemma}
\begin{proof}
	设 $\frac{f}{g} = \frac{f_1}{g_1}$,则 $fg_1 = gf_1$,从而 $\deg f + \deg g_1 = \deg g + \deg f_1$. 因此
	\[
		\deg f - \deg g = \deg f_1 - \deg g_1. \qedhere
	\]
\end{proof}
因此把 $\deg f - \deg g$ 称为分式 $\frac{f}{g}$ 的次数. 分式 $\frac{0}{1}$ 的次数为 $-\infty$. 

类似于一元分式域的构造方法,我们还可以构造出 $R$ 上的 $n$ 元分式域. 
\subsection{多项式环与多项式函数}
按照多项式的加法和乘法的具体定义,当下看出

\[
(f+g)(x,y,\ldots) = f(x,y,\ldots) + g(x,y,\ldots),
\]
\[
(fg)(x,y,\ldots) = f(x,y,\ldots)g(x,y,\ldots),
\]
(常数多项式 $c)(x,y,\ldots) = c$.

因此每个多项式 $f \in R[X,Y,\ldots]$ 都确定从 $R \times R \times \cdots$ (乘积项数 = 变元个数) 到 $R$ 的映射,这是多项式 $f$ 所确定的多项式函数。

\textbf{例 3.3.5} 对于一般的交换环 $R$,多项式未必由它对应的多项式函数确定。一个例子是取 $R = \mathbb{F}_p := \mathbb{Z}/p\mathbb{Z}$,其中 $p$ 是素数。根据 Fermat 小定理 2.8.6,单变元多项式

\[
f(X) = X^p - X \in \mathbb{F}_p[X]
\]

对所有 $x \in \mathbb{F}_p$ 都满足 $f(x) = 0$,所以尽管 $X^p - X$ 并非零多项式,它作为多项式函数却无异于零函数。推而广之,对于任意有限域 $F$,非零多项式 $f(X) := \prod_{a \in F} (X - a)$ 在任何 $a \in F$ 上取值皆为 0。

有鉴于此,对于一般的交换环,必须区分作为一个代数表达式的多项式以及相应的函数或映射,前者才是第一义的。我们将在 \S3.6 说明何时可以等同一个多项式及它所对应的函数。
\subsection{域的特征}
我们用$0_R$代表环 $ R $ 的零元,用$1_R$代表$R$的幺元,作为区分. 

我们常见的域(如 $ \mathbb{Q}, \mathbb{R}, \mathbb{C} $ )中,成立\begin{equation}
	(\forall )n \cdot 1_F = 0_F \iff n = 0,
\end{equation}  而域 $ F = \mathbb{F}_p $ 中成立\begin{equation}\label{field-char-2}
p \cdot 1_F = 0_F; 
\end{equation}
这又蕴涵了对于任意 $ x \in F $ 都有 $ px = (p \cdot 1_F) \cdot x = 0_F $. 

性质\eqref{field-char-2}并非有限域独有. 考虑 $ \mathbb{F}_p $ 上的有理函数域 $ \mathbb{F}_p(X) $ ,它有无穷多个元素,但也满足 $ p \cdot 1_{\mathbb{F}_p(X)} = 0_{\mathbb{F}_p(X)} $,这点只须在其子域 $ \mathbb{F}_p $ 里验证.  
\begin{lemma}{}{field-char}
	对于任意环 $ R $,存在唯一的环同态\begin{align*}
		\mathbb{Z}&\longrightarrow R\\n&\longmapsto n \cdot 1_R. 
	\end{align*}
\end{lemma}
\begin{proof}
	唯一性:注意到环同态必然映$1$为 $ 1_R $,从而映 $ n \geq 0 $ 为 $ 1_R + \cdots 1_R = \underbrace{n \cdot 1_R}_{n \text{ 项}} $,而在 $ n < 0 $ 时映 $ n = -|n| $ 为 $ -(|n| \cdot 1_R) = n \cdot 1_R $. 
	
	存在性问题则归结为检验 $ n \mapsto n \cdot 1_R $ 确实是环同态,这是容易验证的. 
\end{proof}
\begin{definition}{特征}{char}
	设 $ R $ 为整环,若存在唯一的 $ \text{char}(R) \in \mathbb{Z}_{\geq 0} $ 使得对所有 $ n \in \mathbb{Z} $ 都有
	\[ n \cdot 1_R = 0_R \iff \text{char}(R) \mid n, \]称之为整环 $ R $ 的\textbf{特征};它或者是 0,或者是素数. 
\end{definition}
\begin{proof}
	记 $ K_R := \{n \in \mathbb{Z} : n \cdot 1_R = 0_R\} $,它包含$0$,对加法封闭,而且若 $ n \in K_R $ 而 $ m \in \mathbb{Z} $,则 $ mn \cdot 1_R = (m \cdot 1_R)(n \cdot 1_R) = 0_R $ 蕴涵 $ mn \in K_R $. 基于这两种封闭性,引理 \ref{lem:field-char} 遂说明存在唯一的 $ \text{char}(R) \in \mathbb{Z}_{\geq 0} $ 使得 $ K_R = \text{char}(R)\mathbb{Z} $. 设 $ \text{char}(R) \neq 0 $,而且有因数分解 $ \text{char}(R) = ab $,则因为 $ n \mapsto n \cdot 1_R $ 是环同态,故
	\[ \text{char}(R) \cdot 1_R = (a \cdot 1_R)(b \cdot 1_R) = 0_R. \]
	又因为 $ R $ 是整环,必有 $ a \in K_R $ 或 $ b \in K_R $,因此必有 $ \text{char}(R) \mid a $ 或 $ \text{char}(R) \mid b $;留意到 $ \text{char}(R) \neq 1 $(否则将有 $ 1_R = 0_R $). 这足以说明 $ \text{char}(R) $ 若非零则必为素数. 
	
	因此在特征为 $ p > 0 $ 的整环 $ R $ 中,任意 $ x \in R $ 的 $ p $ 倍必然为零:$ px = (p \cdot 1_R)x = 0_R x = 0_R $. 
\end{proof}
\begin{example}{}{}
	设 $ p $ 为素数,而 $ R $ 为满足 $ p \cdot 1_R = 0_R $ 的交换环(例如特征 $ p $ 的整环),则对所有 $ x, y \in R $ 皆有
	\[ (x + y)^p = x^p + y^p. \]
\end{example}
\begin{proof}
	利用二项式定理,只需证明二项式系数 $ \binom{p}{k} $ 在 $ 0 < k < p $ 时总是 $ p $ 的倍数. 注意到\[p\cdot\frac{(p-1)!}{(k-1)!(p-k)!}=k\cdot\frac{p!}{k!(p-k)!}=k\binom{p}{k},\]且$(k,p)=1$,这就证明了$p\mid\binom{p}{k}.$
\end{proof}
\begin{proposition}{}{}
	若 $ R_0 $ 是整环 $ R $ 的子环,则 $ \text{char}(R_0) = \text{char}(R) $. 
\end{proposition}
\begin{proof}
	本书规定子环 $ R_0 $ 必满足 $ 1_R = 1_{R_0} $,所以等式 $ n \cdot 1_R = 0_R $ 成立与否可以在子环 $ R_0 $ 中判定. $\square$
\end{proof}

整环 $ R $ 的特征和它的分式域的特征是一回事:诚然,根据命题 3.7.4,从 $ R \subset \text{Frac}(R) $ 可见 $ \text{char}(R) = \text{char}(\text{Frac}(R)) $. 
\begin{proposition}{}{}
	设 $ E $ 和 $ F $ 为域,$ \text{char}(E) \neq \text{char}(F) $,证明:不存在从 $ E $ 到 $ F $ 的环同态. 
\end{proposition}
\begin{proof}
	利用命题\ref{pro:ring homo injection}. 
\end{proof}
因此,不同特征的域无法直接沟通,除非通过一个较大的整环相联系,例如

\[ \mathbb{F}_p \xleftarrow{\text{商映射}} \mathbb{Z} \xleftarrow{\text{包含}} \mathbb{Q}, \]

或者是运用更复杂的代数或数论技术. 
\subsection{理想}
\subsubsection{理想}
\begin{definition}{理想}{ideal}
	环$R$的一个理想$I$是$R$的满足下列性质的非空子集\footnote{不同的教材对理想的定义也不太一样,与环的定义有一定的关系. }:
	\begin{itemize}
		\item $I$ 在加法下封闭. 
		\item 如果$s \in I, \, r \in R,$ 则$rs \in I$,并且$s \in I, \, t \in R,$ 则$st \in I$. 
	\end{itemize}
\end{definition}
\begin{remark}
	由于定义中不要求 $R$ 交换,所以乘法封闭性必须对双边来陈述;对于非交换环,我们也经常将上述定义中的理想称为 $R$ 的 \textbf{双边理想}. 从上述定义中容易给出\textbf{左理想}、\textbf{右理想}的定义. 本笔记所指理想均为\textbf{双边理想}. 
\end{remark}
\textbf{理想自动对加法逆元封闭}:若 $x \in I$ 则 $-x \in I$,这是基于环论的等式 $-x = (-1) \cdot x$ 和理想的乘法封闭性. 理想的平凡例子有 $I = \{0\}$(零理想)和 $I = R$. 满足 $I \neq R$ 的理想 $I$ 称为\textbf{真理想}. 
\begin{proposition}{}{}
	设$I$是$R$的理想,则$I=R\iff 1\in I.$
\end{proposition}
\begin{proof}
	“$\implies$”显然. “$\Longleftarrow$”. 若$1\in I$,则$\forall r\in R, r=1\cdot r\in I.$
\end{proof}
这表明,真理想不可能是$R$的子环,因为它不含乘法幺元$1$. 
\begin{example}{整数环的理想}{}
	整数环 $\mathbb{Z}$ 的任一子环必形如 $m\mathbb{Z}$, $m \geqslant 0$. 容易用理想的定义验证 $m\mathbb{Z}$ 是 $\mathbb{Z}$ 的理想,因此 $m\mathbb{Z}$, $m \geqslant 0$ 也是 $\mathbb{Z}$ 所有的理想. 
\end{example}
\begin{example}{函数环的理想}{}
	考虑 $C(\mathbb{R})$. 取定 $x_0 \in \mathbb{R}$,定义
	\[ Z_{x_0}(\mathbb{R}) = \{ f \in C(\mathbb{R}) \mid f(x_0) = 0 \}, \]
	则 $Z_{x_0}(\mathbb{R})$ 是 $C(\mathbb{R})$ 的理想. 
	
	函数环的另一个理想在微分几何中有重要作用. 设 $x$ 为 $\mathbb{R}^n$ 中的一点,在 $C^{\infty}(\mathbb{R}^n)$ 中我们定义
	\[ O_x = \{ f \in C^{\infty}(\mathbb{R}^n) \mid \text{存在 } x \text{ 的一个邻域 } U, \text{ 使得 } f(y) = 0, \, \forall y \in U \}. \]
	则容易验证 $O_x$ 是 $C^{\infty}(\mathbb{R}^n)$ 的一个理想. 
\end{example}
给定了理想,如何构造新的理想呢?容易证明一个环的任意多个理想之交仍为理想. 现在设 $S$ 为环 $R$ 的非空子集,则 $R$ 中所有包含 $S$ 的理想(这样的理想是存在的,例如 $R$ 本身就是一个)之交仍为 $R$ 的理想,称为由 $S$ 生成的理想,记为 $\langle S \rangle$. 我们断言 $\langle S \rangle$ 是 $R$ 中包含集合 $S$ 的最小理想. 事实上,由上面的定义,$\langle S \rangle$ 是理想,且包含 $S$. 另一方面,因为 $\langle S \rangle$ 是所有包含 $S$ 的理想之交,因此任何包含 $S$ 的理想一定包含 $\langle S \rangle$,因此 $\langle S \rangle$ 是最小的. 
\begin{example}{包含理想的最小理想}{}
	我们证明
	\[
	\langle S \rangle = \left\{ \sum_{i=1}^n x_i a_i \bigg| n \in \mathbb{N}, x_i \in R, a_i \in S, i = 1, 2, \cdots, n \right\}.
	\]
	事实上,将上式右边的集合记为 $I$. 则对任何 $a \in S, \, a = 1 \cdot a \in I$($1$ 为 $R$ 的幺元),故 $S \subseteq I$. 又由命题 2.2.11 容易看出 $I$ 是 $R$ 的理想. 另一方面,若 $I_1$ 为 $R$ 的一个理想且包含 $S$,则对任何 $x_i \in R$,以及 $a_i \in S, 1 \leqslant i \leqslant n$,有 $x_i a_i \in I_1$,故 $\sum x_i a_i \in I_1$. 故 $I \subseteq I_1$,这说明 $I$ 是包含 $S$ 的最小理想,因此 $I = \langle S \rangle$. 
\end{example}
\begin{definition}{主理想与生成元}{}
	设 $I$ 为环 $R$ 的理想,如果存在 $a \in I$ 使得 $I = \langle a \rangle$,则称 $I$ 为主理想,而 $a$ 称为 $I$ 的一个生成元. 
\end{definition}
\begin{proposition}{环同态的核}{}
	设 $f: R \rightarrow R'$ 为环同态,其核(又称零核)定义为
	\[
	\ker(f) := f^{-1}(0) = \{ x \in R : f(x) = 0 \}.
	\]
	这是 $R$ 的理想. 
\end{proposition}
\begin{proof}
	首先验证加法封闭:若 $x, y \in \ker(f)$,则 $f(x + y) = f(x) + f(y) = 0 + 0 = 0$,故 $x + y \in \ker(f)$. 其次验证乘法双边封闭. 若 $x \in \ker(f)$ 而 $r \in R$,则
	\[
	f(xr) = f(x)f(r) = 0 \cdot f(r) = 0 = f(r) \cdot 0 = f(r)f(x) = f(rx),
	\]
	因此 $xr, rx \in \ker(f)$. 
\end{proof}
\subsubsection{商环}

\section{模}
\begin{definition}{模}{}
	所谓左$R$-模,是指\begin{enumerate}
		\item 加法群;
		\item 映射$R\times M\to M$,以乘法记号记为$(r,m)\mapsto r\cdot m=rm$,也称为模的纯量乘法,满足如下条件:\begin{itemize}
			\item $r(m_1+m_2)=rm_1+rm_2,$
			\item $(r_1+r_2)m=r_1m+r_2m,$
			\item $(r_1r_2)m=r_1(r_2m),$
			\item $1_Rm=m.$
		\end{itemize}
		其中$r_1,r_2\in R$而$m,m_i(i=1,2)\in M.$
		
	\end{enumerate}
	类似可以定义右$R$-模.  
\end{definition}
可以看出,模是线性空间定义的推广,相当于环上的线性空间. 

\section{有理标准形}
\subsection{线性映射和模结构}
\subsection{有理标准形}
\newpage
\begin{thebibliography}{99}  
	
	\bibitem{1}谢启鸿,姚慕生,吴泉水. 高等代数学(第四版). 上海:复旦大学出版社, 2022.
	\bibitem{2}谢启鸿,姚慕生. 高等代数(第四版). 上海:复旦大学出版社, 2022.
	\bibitem{3}丘维声. 高等代数(第二版,上册). 北京:清华大学出版社, 2018.  上海:复旦大学出版社, 2022.
	\bibitem{4}丘维声. 高等代数(第二版,下册). 北京:清华大学出版社, 2018. 
	\bibitem{5}邓少强,朱富海. 抽象代数. 北京:科学出版社,2017. 
	\bibitem{6}李文威. 代数学讲义. 网络版(编译日期: 2025-04-04),来自\url{https://www.wwli.asia/downloads/books/EAlg-Notes.pdf}
	
\end{thebibliography}
\end{document}
