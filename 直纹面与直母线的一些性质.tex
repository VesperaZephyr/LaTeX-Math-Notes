\documentclass[12pt, a4paper,newtx]{ctexart}
\usepackage{amsmath, amsthm, amssymb, appendix, bm, graphicx, hyperref, mathrsfs, geometry}
\geometry{left=2cm,right=2cm,top=2cm,bottom=2cm}
\usepackage{indentfirst}
\setlength{\parindent}{2em}	%设置首行缩进
\usepackage{manfnt}
\usepackage{mathptmx}%修改字体
\usepackage{fancyhdr} %设置页眉页脚
\renewcommand{\footrulewidth}{0.5pt}
\pagestyle{fancy}
\fancyhf{}	%清除页眉页脚R
\fancyfoot[C]{\thepage}
\fancyhead[L]{\kaishu \leftmark}
\fancyhead[C]{\fangsong 关注作者知乎\href{https://www.zhihu.com/people/11-68-36-12}{\heiti 晨锦辉永生之语}谢谢喵}
\fancyhead[R]{\textit\thepage}

\allowdisplaybreaks[1] %n的值为0到4,表示分页的坚决程度,例如0表示能不分页就不分页,4表示强制分页. 
\title{\textbf{直纹面与直母线的一些性质}}
\author{刘晨希}
\date{\today}
\linespread{1.5}

\usepackage{tcolorbox}
\tcbuselibrary{most}
\newtcbtheorem[number within=section]{theorem}{定理}%
{colback=green!5,colframe=green!50!black,fonttitle=\bfseries}{thm}
\newtcbtheorem[number within=section]{definition}{定义}%
{colback=orange!5,colframe=orange!70!black,fonttitle=\bfseries}{def}
\newtcbtheorem[number within=section]{conclusion}{结论}%
{colback=cyan!5,colframe=cyan!70!black,fonttitle=\bfseries}{con}
\newtcbtheorem[number within=section]{example}{例}%
{colback=purple!5,colframe=purple!80!black,fonttitle=\bfseries}{ex}
\newtcbtheorem[number within=section]{property}{性质}%
{colback=yellow!5,colframe=yellow!35!green,fonttitle=\bfseries}{proper}
\newenvironment{remark}{\dbend\textbf{注. }}{}{}
\newenvironment{hint}{\textbf{答案或提示. }}{}{}

\begin{document}
	
	\maketitle
	温馨提示:本文档的公式可以点击跳转,点击公式标号红方框即可实现公式跳转(试一试!),点击页眉的蓝色方框有可能实现跳转到作者知乎首页(与pdf阅读器有关,QQ手机自带的pdf在线浏览可能不太支持),如果能点个关注就更好了,谢谢喵. 
\section{直纹面}
\begin{definition}{直纹面与直母线}{}\kaishu 
	一曲面$S$称为\textbf{直纹面},如果存在一族直线,使得这一族中的每一条直线全在$S$上,并且$S$上的每个点都在这一族的某一条直线上. 这样的一族直线称为$S$的\textbf{一族直母线}.
\end{definition}
\begin{example}{直纹面的例子}{}
	二次柱面(9种)和二次锥面(1种)都是直纹面;单叶双曲面和双曲抛物面都是直纹面. 
\end{example}
下面我们来证明单叶双曲面和双曲抛物面都是直纹面. 
\begin{theorem}{}{}
	单叶双曲面和双曲抛物面都是直纹面. 
\end{theorem}
我们只证明单叶双曲面的一种情形,其余情况都是类似的,留作读者习题. 
\begin{proof}
	设单叶双曲面的方程为\begin{equation}
		\dfrac{x^2}{a^2}+\dfrac{y^2}{b^2}-\dfrac{z^2}{c^2}=1.
	\end{equation}
	点$M_0(x_0,y_0,z_0)^T$在单叶双曲面$S$上的充分必要条件是\begin{equation}
		\dfrac{x_0^2}{a^2}+\dfrac{y_0^2}{b^2}-\dfrac{z_0^2}{c^2}=1.
	\end{equation}
	移项并分解因式,得到\begin{equation}
		\left(\dfrac{x_0}a+\dfrac{z_0}{c}\right)\left(\dfrac{x_0}{a}-\dfrac{z_0}{c}\right)=\left(1+\dfrac{y_0}{b}\right)\left(1-\dfrac{y_0}{b}\right),
	\end{equation}
	写成行列式的形式,即\begin{equation}\label{eq4}
			\begin{vmatrix}
			\dfrac{x_0}{a}+\dfrac{z_0}{c}&1+\dfrac{y_0}{b}\\1-\dfrac{y_0}{b}&\dfrac{x_0}{a}-\dfrac{y_0}{c}
		\end{vmatrix}=0,
	\end{equation}
	或者\begin{equation}\label{eq5}
		\begin{vmatrix}
			\dfrac{x_0}{a}+\dfrac{z_0}{c}&1-\dfrac{y_0}{b}\\1+\dfrac{y_0}{b}&\dfrac{x_0}{a}-\dfrac{y_0}{c}
		\end{vmatrix}=0. 
	\end{equation}
	注意到$1+\dfrac{y_0}{b},1-\dfrac{y_0}{b}$不全为零,故方程组\begin{equation}\label{equations}
		\begin{cases}
			\left(\dfrac{x_0}a+\dfrac{z_0}{c}\right)X+\left(1+\dfrac{y_0}{b}\right)Y=0,\\\left(1+\dfrac{y_0}{b}\right)X+\left(\dfrac{x_0}a-\dfrac{z_0}{c}\right)Y=0
		\end{cases}
	\end{equation}
	是关于$X,Y$的齐次线性方程组,它的系数矩阵的行列式为零\eqref{eq4},故有非零解,即存在不全为零的实数$\mu_0,\nu_0$,使得
	\begin{equation}
		\begin{cases}
			\mu_0\left(\dfrac{x_0}a+\dfrac{z_0}{c}\right)+\nu_0\left(1+\dfrac{y_0}{b}\right)=0,\\\mu_0\left(1-\dfrac{y_0}{b}\right)+\nu_0\left(\dfrac{x_0}a-\dfrac{z_0}{c}\right)=0.
		\end{cases}
	\end{equation}
	这表明点$M_0$在直线\begin{equation}
		\begin{cases}
			\mu_0\left(\dfrac{x}a+\dfrac{z}{c}\right)+\nu_0\left(1+\dfrac{y}{b}\right)=0,\\\mu_0\left(1-\dfrac{y}{b}\right)+\nu_0\left(\dfrac{x}a-\dfrac{z}{c}\right)=0.
		\end{cases}
	\end{equation}
	上. 也就是\textbf{单叶双曲面}$S$上的任一点$M_0$在直线族
	\begin{equation}\label{ruled surface}
		\begin{cases}
			\mu\left(\dfrac{x}a+\dfrac{z}{c}\right)+\nu\left(1+\dfrac{y}{b}\right)=0,\\\mu\left(1-\dfrac{y}{b}\right)+\nu\left(\dfrac{x}a-\dfrac{z}{c}\right)=0,
		\end{cases}\quad\text{其中}\mu,\nu\in\mathbb R,\mu^2+\nu^2\ne0
	\end{equation}上. 我们断言:这族直线\eqref{ruled surface}恰为单叶双曲面的一族直母线. 
	
	事实上,若$(\mu_1,\nu_1),(\mu_2,\nu_2)$线性相关,则它们都表示\eqref{ruled surface}中的一条直线;若它们线性无关,则它们表示不同的直线. 即$\mu,\nu$的比例决定了\eqref{ruled surface}. 现在我们再\eqref{ruled surface}中任取一条直线$l_1,$它对应于$(\mu_1,\nu_1)$,且在$l_1$上任取一点$M_1(x_1,y_1,z_1)^T$,则有\begin{equation}\label{eq10}
		\begin{cases}
			\mu_1\left(\dfrac{x_1}a+\dfrac{z_1}{c}\right)+\nu_1\left(1+\dfrac{y_1}{b}\right)=0,\\\mu_1\left(1-\dfrac{y_1}{b}\right)+\nu_1\left(\dfrac{x_1}a-\dfrac{z_1}{c}\right)=0.
		\end{cases}
	\end{equation}
	因为$\mu_1,\nu_1$不全为零,则\eqref{eq10}说明齐次线性方程组\begin{equation}\label{eq11}
		\begin{cases}
			\left(\dfrac{x_1}a+\dfrac{z_1}{c}\right)X+\left(1+\dfrac{y_1}{b}\right)Y=0,\\\left(1-\dfrac{y_1}{b}\right)X+\left(\dfrac{x_1}a-\dfrac{z_1}{c}\right)Y=0
		\end{cases}
	\end{equation}
	有非零解,从而\eqref{eq11}的系数矩阵的行列式为0,也就是满足\eqref{eq4},即$M_1(x_1,y_1,z_1)^T$在单叶双曲面$S$上. 综上所述,$S$是直纹面,且直线族\eqref{ruled surface}是它的一族直母线. 
	
	类似可证$S$的另一族直母线为\begin{equation}
		\label{ruled surface2}
		\begin{cases}
			\mu\left(\dfrac{x}a+\dfrac{z}{c}\right)+\nu\left(1-\dfrac{y}{b}\right)=0,\\\mu\left(1+\dfrac{y}{b}\right)+\nu\left(\dfrac{x}a-\dfrac{z}{c}\right)=0,
		\end{cases}\quad\text{其中}\mu,\nu\in\mathbb R,\mu^2+\nu^2\ne0.
	\end{equation}
\end{proof}
我们总结出如下结论,需要记忆:\begin{conclusion}{单叶双曲面的直母线方程}{}
	设单叶双曲面$S:\dfrac{x^2}{a^2}+\dfrac{y^2}{b^2}-\dfrac{z^2}{c^2}=1$,则它的两族直母线方程为
	\begin{equation}\label{1.1.1}
		\begin{cases}
			\mu\left(\dfrac{x}a+\dfrac{z}{c}\right)+\nu\left(1+\dfrac{y}{b}\right)=0,\\\mu\left(1-\dfrac{y}{b}\right)+\nu\left(\dfrac{x}a-\dfrac{z}{c}\right)=0,
		\end{cases}\quad\text{其中}\mu,\nu\in\mathbb R,\mu^2+\nu^2\ne0\tag{$\spadesuit$}
	\end{equation}
	和\begin{equation}\label{1.1.2}
		\begin{cases}
			\mu\left(\dfrac{x}a+\dfrac{z}{c}\right)+\nu\left(1-\dfrac{y}{b}\right)=0,\\\mu\left(1+\dfrac{y}{b}\right)+\nu\left(\dfrac{x}a-\dfrac{z}{c}\right)=0,
		\end{cases}\quad\text{其中}\mu,\nu\in\mathbb R,\mu^2+\nu^2\ne0.\tag{$\heartsuit$}
	\end{equation}
\end{conclusion}
\begin{conclusion}
	{双曲抛物面的直母线方程}{}
	设双曲抛物面$\Gamma:\dfrac{x^2}{a^2}-\dfrac{y^2}{b^2}=2z$,则它的两族直母线方程为
	\begin{equation}\label{1.2.1}
		\begin{cases}
			\left(\dfrac{x}a+\dfrac{y}{b}\right)+2\lambda=0,\\
			z+\lambda\left(\dfrac{x}a-\dfrac{y}{b}\right)=0,
		\end{cases}\quad\text{其中}\lambda\in\mathbb{R}\tag{$\clubsuit$}
	\end{equation}
	和\begin{equation}\label{1.2.2}
		\begin{cases}
			\lambda\left(\dfrac{x}a+\dfrac{y}{b}\right)+z=0,\\
			2\lambda+\left(\dfrac{x}a-\dfrac{y}{b}\right)=0,
		\end{cases}\quad\text{其中}\lambda\in\mathbb{R}.\tag{$\diamondsuit$}
	\end{equation}
\end{conclusion}
下面我们来看几个例题.
\begin{example}{}{}
	求单叶双曲面$\dfrac{x^2}{4}+\dfrac{y^2}{9}-\dfrac{z^2}{16}=1$的经过点$(2,3,-4)^T$的直母线. 
\end{example}
\begin{hint}
	\[\begin{cases}
		2x+z=0,\\y-3=0;
	\end{cases}\quad\begin{cases}\dfrac{x}{2}+\dfrac{z}{4}+\dfrac{y}{3}-1=0,\\\dfrac{y}{3}-\dfrac{x}{2}+\dfrac{z}{4}+1=0.
	\end{cases}\]
\end{hint}
\begin{example}{}{}
	求与下列三条直线同时共面的直线所构成的曲面:\[l_1:\begin{cases}
		x=1,\\y=z;
	\end{cases}\quad l_2:\begin{cases}
	x=-1,\\y=-z;
	\end{cases}\quad l_3:=\dfrac{x-2}{-3}=\dfrac{y+1}{4}=\dfrac{z+2}{5}.\]
\end{example}
\begin{hint}
	设与$l_1,l_2,l_3$同时共面的直线$l$的方程为\[\frac{x-x_0}{X}=\dfrac{y-y_0}{Y}=\dfrac{z-z_0}{Z}.\]注意题目条件$l$与$l_i(i=1,2,3)$共面,根据直线共面的充要条件,列出三个方程. 把它们看作关于$X,Y,Z$的方程. $X,Y,Z$一定存在且不全为零,则线性方程组的系数矩阵的行列式等于0,由此得到$x_0,y_0,z_0$满足的方程,它就是所求曲面的方程. 最后算得$x^2+y^2-z^2=1$,它是单叶双曲面. 
\end{hint}
\begin{example}{}{}
	设有直线$l_1,l_2$,它们的方程分别是\[\begin{cases}
		x=\dfrac{3}{2}+3t,\\y=-1+2t,\\z=-t,
	\end{cases}\quad\begin{cases}
	x=3t,\\y=2t,\\z=0,
	\end{cases}\]求所有由$l_1,l_2$上有相同参数值$t$的点的连线所构成的曲面的方程. 
\end{example}
\begin{hint}
	任意给定$t_0\in\mathbb{R}$,则连线经过两点$M_1\left(\dfrac{3}{2}+3t_0,-1+2t_0,-t_0\right)^T,M_2\left(3t_0,2t_0,0\right)^T$,从而不难写出连线的方程\[\frac{x-3t_0}{\dfrac32}=\dfrac{y-2t_0}{-1}=\dfrac{z}{-t_0}.\]消去参数$t_0$,就得到$\dfrac{y^2}{4}-\dfrac{x^2}{9}=2z$,它是双曲抛物面. 
\end{hint}
\section{直母线的性质}
\subsection{需要的结论}
这里我们首先列举几个在后面证明中常用的结论:\begin{conclusion}{两条直线相交的充分必要条件}{intersect}
	设两条直线$l_1,l_2$分别过$M_1\left(x_1,y_1,z_1\right)^T,M_2\left(x_2,y_2,z_2\right)^T$,它们的方向向量分别为$\bm u_1=\left(X_1,Y_1,Z_1\right)^T,\bm u_2=\left(X_2,Y_2,Z_2\right)^T$. 则这两条直线相交的充分必要条件为 $\bm{u}_1, \bm{u}_2, \overrightarrow{M_1M_2}$ 共面,且 $\bm{u}_1, \bm{u}_2$ 不共线,即判别式
	\[
	\Delta := \begin{vmatrix}
		x_2 - x_1 & X_1 & X_2 \\
		y_2 - y_1 & Y_1 & Y_2 \\
		z_2 - z_1 & Z_1 & Z_2
	\end{vmatrix} = 0 \, .\]
\end{conclusion}
\begin{conclusion}{两条直线异面的充分必要条件}{not intersect}
	设两条直线$l_1,l_2$分别过$M_1\left(x_1,y_1,z_1\right)^T,M_2\left(x_2,y_2,z_2\right)^T$,它们的方向向量分别为$\bm u_1=\left(X_1,Y_1,Z_1\right)^T,\bm u_2=\left(X_2,Y_2,Z_2\right)^T$. 则这两条直线相交的充分必要条件为 $\bm{u}_1, \bm{u}_2, \overrightarrow{M_1M_2}$ 不共面,且 $\bm{u}_1, \bm{u}_2$ 不共线,即判别式
	\[
	\Delta := \begin{vmatrix}
		x_2 - x_1 & X_1 & X_2 \\
		y_2 - y_1 & Y_1 & Y_2 \\
		z_2 - z_1 & Z_1 & Z_2
	\end{vmatrix} \ne 0 \, .\]
\end{conclusion}
\begin{conclusion}{直线和平面相交的充要条件}{line and space}
	设直线$l$的方向向量为$\bm v=(X,Y,Z)^T$,平面$\pi$的方程为$Ax+By+Cz+D=0$,则$l$与$\pi$相交的充分必要条件是$\bm v$与$\pi$不平行,即\[AX+BY+CZ\ne0.\]
\end{conclusion}
\subsection{基础的性质}
\begin{property}{}{}
	单叶双曲面同族中的任意三条直母线都不平行于同一个平面. 
\end{property}
\begin{proof}
	我们取单叶双曲面的方程$\dfrac{x^2}{a^2} + \dfrac{y^2}{b^2} - \dfrac{z^2}{c^2} = 1$
	和一族直母线\eqref{1.1.1},
	其方向向量为
	\[
	\bm{w} = \begin{pmatrix}
		a(\mu^2 - \nu^2) \\
		2b\mu\nu \\
		-c(\mu^2 + \nu^2)
	\end{pmatrix},
	\]
	我们分别对 \((\mu_i, \nu_i)(i = 1, 2, 3)\) 赋不同的值,得到直线 \(l_i\) 的方向向量 \(\bm{w}_i\),容易验证 \(\bm{w}_1 \times \bm{w}_2 \cdot \bm{w}_3 \neq 0 \Longrightarrow l_1, l_2, l_3\) 不平行于同一平面. 类似地,可讨论另一族直母线. 
\end{proof}
\begin{property}{}{}
	单叶双曲面同族的两条直母线异面,单叶双曲面异族的两条直母线共面. 
\end{property}
\begin{proof}
	我们取单叶双曲面的方程$\dfrac{x^2}{a^2} + \dfrac{y^2}{b^2} - \dfrac{z^2}{c^2} = 1$,再取\eqref{1.1.1}中的两条直线\[\ell_1:\begin{cases}
		\mu_1\left(\dfrac{x}a+\dfrac{z}{c}\right)+\nu_1\left(1+\dfrac{y}{b}\right)=0,\\\mu_1\left(1-\dfrac{y}{b}\right)+\nu_1\left(\dfrac{x}a-\dfrac{z}{c}\right)=0,
	\end{cases}\quad \ell_2:\begin{cases}
	\mu_2\left(\dfrac{x}a+\dfrac{z}{c}\right)+\nu_2\left(1+\dfrac{y}{b}\right)=0,\\\mu_2\left(1-\dfrac{y}{b}\right)+\nu_2\left(\dfrac{x}a-\dfrac{z}{c}\right)=0.
	\end{cases}\]
	容易算得它们的方向向量分别为\begin{equation}\label{eq13}
		\bm u_1=\left(a(\mu_1^2 - \nu_1^2),
		2b\mu_1\nu_1,-c(\mu_1^2 + \nu_1^2)\right)^T,\bm u_2=\left(a(\mu_2^2 - \nu_2^2),
		2b\mu_2\nu_2,-c(\mu_2^2 + \nu_2^2)\right)^T.\
	\end{equation}
	两条直线异面的充分必要条件(结论\ref{con:not intersect}),现在分别取直线$\ell_1,\ell_2$上一点$M_1,M_2$,再计算判别式知其不为0\footnote{请自行完成,留作习题},故这两条直线异面.
	 
	取\eqref{1.1.1}中的两条直线\[l_1:\begin{cases}
		\mu_1\left(\dfrac{x}a+\dfrac{z}{c}\right)+\nu_1\left(1+\dfrac{y}{b}\right)=0,\\\mu_1\left(1-\dfrac{y}{b}\right)+\nu_1\left(\dfrac{x}a-\dfrac{z}{c}\right)=0,
	\end{cases}\]前面\eqref{eq13}
	我们已经计算出其方向向量:$\bm{v}_1=\left(a(\mu_1^2 - \nu_1^2),
	2b\mu_1\nu_1,-c(\mu_1^2 + \nu_1^2)\right)^T$,再取\eqref{1.1.2}中的一条直线\[l_2:\begin{cases}
	\mu_2\left(\dfrac{x}a+\dfrac{z}{c}\right)+\nu_2\left(1-\dfrac{y}{b}\right)=0,\\\mu_2\left(1+\dfrac{y}{b}\right)+\nu_2\left(\dfrac{x}a-\dfrac{z}{c}\right)=0,
	\end{cases}\]
	它的方向向量为$\bm{v}_2=\left(-a(\mu_2^2 - \nu_2^2),
	2b\mu_2\nu_2,c(\mu_2^2 +l \nu_2^2)\right)^T.$现在分别取直线$l_1,l_2$上一点$L_1,L_2$,再计算判别式知其为0\footnote{请自行完成,留作习题},故这两条直线共面.
\end{proof}
\begin{property}{}{2.3}
	双曲抛物面同族的所有直母线都平行于同一个平面,并且同族的任意两条直母线异面. 
\end{property}
\begin{proof}
	取双曲抛物面的方程:$\dfrac{x^2}{a^2}-\dfrac{y^2}{b^2}=2z$和它的两族直母线方程\eqref{1.2.1}和\eqref{1.2.2},容易看出对于第一族直线\eqref{1.2.1},它们的方向向量均为\[\left(\dfrac1b,-\dfrac1a,-\dfrac{2\lambda}{ab}\right)^T,\]故它们均平行于平面$\pi_0:\dfrac{x}a+\dfrac{y}b=0$,进一步地,可以计算出它们都在平面\[\pi_{\lambda}:\frac{x}{a}+\frac{y}{b}+2\lambda=0\]上. 另一族同理(方向向量计算留作习题). 
	
	另一方面,任取\eqref{1.2.1}中两条直线$\ell_{\lambda_1},\ell_{\lambda_2}(\lambda_1\ne\lambda_2)$,则它们在不同的平行平面$\pi_{\lambda_1},\pi_{\lambda_2}$上,故它们不相交;由于$\left(\dfrac1b,-\dfrac1a,-\dfrac{2\lambda_1}{ab}\right)^T\ne\left(\dfrac1b,-\dfrac1a,-\dfrac{2\lambda_2}{ab}\right)^T$,故$\ell_{\lambda_1}$与$\ell_{\lambda_2}$不平行,故它们异面. 
\end{proof}
\begin{property}{}{2.4}
	双曲抛物面异族的任意两条直母线必相交. 
\end{property}
\begin{proof}
	取双曲抛物面的方程:$\dfrac{x^2}{a^2}-\dfrac{y^2}{b^2}=2z$,再取\eqref{1.2.1}中的一条直线\[\ell_1:\begin{cases}
	\left(\dfrac{x}a+\dfrac{y}{b}\right)+2\lambda_1=0,\\
	z+\lambda_1\left(\dfrac{x}a-\dfrac{y}{b}\right)=0,
	\end{cases}\]取\eqref{1.2.2}中的一条直线\[\ell_2:\begin{cases}
	\lambda_2\left(\dfrac{x}a+\dfrac{y}{b}\right)+z=0,\\
	2\lambda_2+\left(\dfrac{x}a-\dfrac{y}{b}\right)=0,
	\end{cases}\]
	容易算得它们的方向向量分别为$\bm u_1=\left(a,-b,-2\lambda_1\right)^T,\bm u_2=\left(a,-b,-2\lambda_2\right)^T.$回忆两条直线相交的充分必要条件(\ref{con:intersect})
	具体地,直线 $l_1$ 经过点 $M_1 \left( -\lambda_1 a, -\lambda_1 b, 0 \right)^T$;直线 $l_2$ 经过点 $M_2 \left( -\lambda_2 a, -\lambda_2 b, 0 \right)^T$.
	
	容易验证 $\overrightarrow{M_1M_2} \cdot \bm{u}_1 \times \bm{u}_2 = 0$,故 $\bm{u}_1, \bm{u}_2, \overrightarrow{M_1M_2}$ 共面;显然 $\bm{u}_1, \bm{u}_2$ 不共线,所以这两条直线相交. 
\end{proof}
\begin{property}{}{}
	双曲抛物面的正交直母线的交点轨迹为双曲线. 
\end{property}
\begin{proof}
	由性质\ref{proper:2.3},双曲抛物面同族的任意两条直母线异面,故双曲抛物面的两条正交直母线必定异族. 由性质\ref{proper:2.4}的证明,我们取相同的双曲抛物面方程,并分别取其方向向量为
	\[
	\bm{u}_1 = \begin{pmatrix}
		a \\
		-b \\
		-2\lambda_1
	\end{pmatrix},\quad\bm{u}_2 = \begin{pmatrix}
	a \\
	b \\
	-2\lambda_2
	\end{pmatrix}
	\]
	那么
	\[
	\bm{u}_1 \cdot \bm{u}_2 = b^2 - a^2 + 4\lambda_1 \lambda_2 = 0 \Longrightarrow \lambda_1 \lambda_2 = \frac{b^2 - a^2}{4}.
	\]
	那么由直母线方程\eqref{1.2.1}和\eqref{1.2.2}\footnote{思考一下怎么得到的}得到
	\[\begin{cases}
		\dfrac{x^2}{a^2} - \dfrac{y^2}{b^2} = 4\lambda_1 \lambda_2 = b^2 - a^2,\\[5pt]\lambda_1 \lambda_2 \left( \dfrac{x^2}{a^2} - \dfrac{y^2}{b^2} \right) = z^2.
	\end{cases}\]
	于是得到方程
	\[
	\begin{cases}
		\dfrac{x^2}{a^2} - \dfrac{y^2}{b^2} = b^2 - a^2, \\[5pt]
		z = \dfrac{b^2 - a^2}{2},
	\end{cases} 
	\]
	这是一条双曲线. 
\end{proof}

\subsection{进一步的结论}
\begin{property}{}{}
	给定单叶双曲面
	\[
	S \colon \frac{x^2}{a^2} + \frac{y^2}{b^2} - \frac{z^2}{c^2} = 1, \quad a, b, c > 0,
	\]
	求经过 $S$ 上一点 $M_0(x_0, y_0, z_0)^T$,沿方向 $(X, Y, Z)^T$ 的直线是 $S$ 的直母线的条件. 由此证明:经过 $S$ 上每一点恰有两条直母线. 
\end{property}
\begin{proof}
	我们写出直线的参数方程
	\begin{equation}\label{eq14}
		l : \begin{cases}
			x = x_0 + Xt, \\
			y = y_0 + Yt, \\
			z = z_0 + Zt,
		\end{cases}
	\end{equation}
	其中 $t$ 是参数,代入方程 $\dfrac{x^2}{a^2} + \dfrac{y^2}{b^2} - \dfrac{z^2}{c^2} = 1$,得到
	\[
	\left( \frac{X^2}{a^2} + \frac{Y^2}{b^2} - \frac{Z^2}{c^2} \right) t^2 + 2 \left( \frac{Xx_0}{a^2} + \frac{Yy_0}{b^2} - \frac{Zz_0}{c^2} \right) t + \frac{x_0^2}{a^2} + \frac{y_0^2}{b^2} - \frac{z_0^2}{c^2} = 0
	\]
	注意 $t$ 的任意性,那么必有
	\[
	\begin{cases}
		\dfrac{X^2}{a^2} + \dfrac{Y^2}{b^2} - \dfrac{Z^2}{c^2} = 0 \\
		\dfrac{Xx_0}{a^2} + \dfrac{Yy_0}{b^2} - \dfrac{Zz_0}{c^2} = 0
	\end{cases}
	\]
	从\eqref{eq14}式容易看出 $Z \neq 0$,故依齐次性,不妨设 $Z = c \neq 0$,两个式子联立,解出$X,Y$即可\footnote{最终结果很丑陋,所以不写出了,不信你试一下},这恰好就是条件. 容易知道$X,Y$有两组,故有两条直母线. 
\end{proof}
\begin{property}{}{}
	单叶双曲面的每条直母线都与腰椭圆相交. 
\end{property}
\begin{proof}
	我们取单叶双曲面的方程$\dfrac{x^2}{a^2} + \dfrac{y^2}{b^2} - \dfrac{z^2}{c^2} = 1$,再取\eqref{1.1.1}中一条直线\[l:\begin{cases}
		\mu_0\left(\dfrac{x}a+\dfrac{z}{c}\right)+\nu_0\left(1+\dfrac{y}{b}\right)=0,\\\mu_0\left(1-\dfrac{y}{b}\right)+\nu_0\left(\dfrac{x}a-\dfrac{z}{c}\right)=0,
	\end{cases}\quad\left(\mu_0^2+\nu_0^2\ne0\right)\]
	它的方向向量为\[\bm u=\begin{pmatrix}
		a(\mu_0^2 - \nu_0^2)\\2b\mu_0\nu_0\\-c(\mu_0^2 + \nu_0^2)
	\end{pmatrix}
	,\]我们只要证明$l$与$xOy$平面相交即可. 注意$xOy$平面的方程$z=0$,根据结论\ref{con:line and space},计算得到直线的方向向量与平面法向量的内积为$-c(\mu_0^2+\nu_0^2)\ne0$,故$l$与$xOy$平面相交即可,这就完成了证明. 
\end{proof}
\begin{property}{}{}
	设 $l_1,l_2$ 是异面直线,它们都与 $Oxy$ 平面相交,证明:与 $l_1,l_2$ 都共面,并且与 $Oxy$ 平面平行的直线所构成的曲面是马鞍面. 
\end{property}
\begin{proof}
	设$l_i(i=1,2)$与$Oxy$平面的交点为$M_i(i=1,2).$ 为简化计算,不妨设原点$O$是线段$M_1,M_2$的中点,令$M_1(a,0,0)^T,M_2(-a,0,0)^T.$ 设$l_i$的方向向量为$\bm v_i=(X_i,Y_i,Z_i)^T$,考虑直线$l$平行于平面$Oxy$,并且与$l_1,l_2$共面,则$l$的方向向量为$\bm v=(X,Y,0)^T.$在$l$上任取一点$M_0(x_0,y_0,z_0)^T$,由$l$与$l_i$共面知\begin{equation}
		\overrightarrow{M_0M_i}\cdot\bm v_i\times\bm v=0,\quad i=1,2.
	\end{equation}
	展开后整理得到\begin{equation}
		\begin{cases}
			X\left(-Z_1y_0+Y_1z_0\right)+Y\left(Z_1x_0-X_1z_0-Z_1a\right)=0,\\X\left(-z_2y_0+Y_2z_0\right)+Y\left(Z_2x_0-X_2z_0+Z_2a\right)=0.
		\end{cases}
	\end{equation}
	由于$X,Y$不全为零,故该齐次线性方程组的系数矩阵的行列式为0,展开得到\begin{equation}\label{eq17}
		\left(X_1Y_2-X_2Y_1\right)z_0^2+\left(Y_1Z_2-Y_2Z_1\right)x_0z_0+\left(X_2Z_1-X_1Z_2\right)y_0z_0-2az_1z_2y_0+a\left(Y_2Z_1+Y_1Z_2\right)z_0=0.
	\end{equation}
	注意$l_1,l_2$异面,由结论\ref{con:not intersect}知判别式\[\Delta:=\begin{vmatrix}
		2a&X_1&X_2\\0&Y_1&Y_2\\0&Z_1&Z_2
	\end{vmatrix}=2a\left(Y_1Z_2-Y_2Z_1\right)\ne0,\]
	即\eqref{eq17}代表的是二次曲面方程,剩下的工作就是进行坐标变换将\eqref{eq17}式化为二次标准型,最终将其化为$\dfrac{x^2}{a^2}-\dfrac{y^2}{b^2}=2z$的形式. 由于计算量较大这里略去,具体的坐标变换方法请参考丘维声《高等代数》(清华大学出版社,339-341页). 
\end{proof}
\begin{property}{}{}
	设三条直线 $l_1$,$l_2$,$l_3$ 两两异面,并且平行于同一平面,证明:与 $l_1$,$l_2$,$l_3$ 都相交的直线所构成的曲面是马鞍面. 
\end{property}
\begin{proof}
	我们以 $l_1$ 为 $x$ 轴,过 $l_1$ 与 $l_2$ 平行的平面为 $Oxy$ 平面,$l_1, l_2$ 的公垂线为 $z$ 轴,建立空间直角坐标系. 故 $l_2$ 的方向向量为 $\bm{v}_2 = (a, 1, 0)^T$,并且经过点 $M_2(0, 0, d)^T$. 设$l_3$ 的方向向量为 $\bm{v}_3 = (b, 1, 0)^T$,并记 $l_3$ 与 $Ozx$ 平面的交点为 $M_3(c, 0, h)^T$,再设直线 $l, l_1$ 交于 $A_1(\lambda, 0, 0)^T$,与 $l_2$ 交于 $A_2(u_1, u_2, d)^T$,与 $l_3$ 交于 $A_3(v_1, v_2, h)^T$,则 $l$ 的方向向量为
	\begin{equation}\label{eq2.9.1}
		\bm{v} = (u_1 - \lambda, u_2, d)^T = (v_1 - \lambda, v_2, h)^T
	\end{equation}
	故$ (u_1 - \lambda, u_2, d) = k(v_1 - \lambda, v_2, h)$,解得\begin{equation}\label{eq2.9.2}
		k = \dfrac{h}{d}.
	\end{equation}
	
	我们写出 $l_1, l_2, l_3$ 的方程:
	\begin{align}
		l_1 : &\quad y = z = 0,\\l_2 : &\quad\frac{x}{a} = \frac{y}{1} = \frac{z - d}{0},\\l_3 : &\quad\frac{x - c}{b} = \frac{y - 1}{1} = \frac{z - h}{0},
	\end{align} 
	将上面三式分别代入 $A_1, A_2, A_3$ 的坐标,得到\begin{equation}\label{eq2.9.6}
		\begin{cases}
			u_1 = a u_2, \\
			v_1 - c = b (v_2 - 1).
		\end{cases} 
	\end{equation}
	结合\eqref{eq2.9.1}和\eqref{eq2.9.2}解得
	\begin{equation}\label{eq24}
		\begin{cases}
			u_1 = a \cdot \dfrac{h(c - b) + (d - h)(b - 1)\lambda}{(a - b)d}, \\
			u_2 = \dfrac{h(c - b) + (d - h)(b - 1)\lambda}{(a - b)d}.
		\end{cases}
	\end{equation}
	从而我们写出 $l$ 的方程
	\[
	\frac{x - \lambda}{a\cdot\dfrac{h(c - b) + (d - h)(b - 1)\lambda}{(a-b)d}-\lambda} = \frac{y}{\dfrac{h(c - b) + (d - h)(b - 1)\lambda}{(a-b)d}} = \frac{z}{d}.
	\]
	若简记$P=\dfrac{h(c - b) + (d - h)(b - 1)\lambda}{(a-b)d}$,消去\eqref{eq24}中的$\lambda$,即解得$\lambda=\dfrac{d(x-ay)}{d-z}$代入\eqref{eq24}中,整理得到\[(x-ay)\left(P(1-d)x+(aP-1)yz+dy\right)=0.\]
	
	再采用如上题相同的方法,先判断该方程表示二次曲面,再将其化为二次标准型,从而判断该曲面是马鞍面. 
\end{proof}

\end{document}